\documentclass[11pt]{article}

\usepackage{amsmath}
\usepackage{amsfonts}
\usepackage{amssymb}

% Give ourself extra space for text
\usepackage[left = 2.2cm, right = 2.2cm, top = 1.8cm, bottom = 2.8cm]{geometry}

% Allows us to easily change the numbering system used in things like \begin{enumerate}. https://ctan.org/tex-archive/macros/latex/contrib/enumitem/
\usepackage[shortlabels]{enumitem}

% Turns table of contents, \refs, etc. into hyperlinks
\usepackage{hyperref}

% To include image, just use \includegraphics[scale=•]{relative path to image}
\usepackage{graphicx}
\graphicspath{ {./images/} }

% Common sets
\newcommand{\integers}{\mathbb{Z}}
\newcommand{\naturals}{\mathbb{N}}
\newcommand{\reals}{\mathbb{R}}
\newcommand{\complex}{\mathbb{C}}


% Power set
\newcommand{\powerset}{\mathcal{P}}

% Inverse hyperbolic functions
\DeclareMathOperator{\arcosh}{arcosh}
\DeclareMathOperator{\arsinh}{arsinh}
\DeclareMathOperator{\artanh}{artanh}

% I hat, J hat, K hat
\newcommand{\ihat}{\boldsymbol{\hat{\textbf{\i}}}}
\newcommand{\jhat}{\boldsymbol{\hat{\textbf{\j}}}}
\newcommand{\khat}{\boldsymbol{\hat{\textbf{k}}}}

% Better vectors (for single characters)
\renewcommand{\vec}[1]{\mathbf{#1}}

% Allows us to number equations in \begin{align} statements, etc.
\newcommand\numberthis{\addtocounter{equation}{1}\tag{\theequation}}

% Augmented matrices: this allows us to make augmented matrics using something like \begin{bmatrix}[cc|c]. Taken from Stefan Kottwitz at https://tex.stackexchange.com/questions/2233/whats-the-best-way-make-an-augmented-coefficient-matrix.
\makeatletter
\renewcommand*\env@matrix[1][*\c@MaxMatrixCols c]{%
  \hskip -\arraycolsep
  \let\@ifnextchar\new@ifnextchar
  \array{#1}}
\makeatother

% NOTE: This means \section does NOT number sections, but ensures that they appear in the table of contents, which does not occur if simply \section* is used. From egreg @ https://tex.stackexchange.com/a/30225.
\setcounter{secnumdepth}{0} % sections are level 1

\begin{document}
\title{ENG1005: Lecture 18}
\author{Lex Gallon}
\maketitle

\tableofcontents

\section*{Video link}
Click \href{https://echo360.org.au/lesson/G_35fe23e0-41ee-4e6f-b0f5-05f4155bb7b0_b944cecf-8ba5-40d3-a870-0243a0a9e78c_2020-04-30T15:58:00.000_2020-04-30T16:53:00.000/classroom#sortDirection=desc}{here} for a recording of the lecture.

\section{Determinants of $n \times n$ matrices - continued}
\subsection{Example}
Compute the determinant of
\[
\begin{bmatrix}
-1 & 2 & -2 \\
2 & 0 & 1 \\
3 & 1 & 2
\end{bmatrix}
\]

\subsection{Solution}
Note: try finding a row with lots of zeroes to make your life easier. Here, we chose row 2.
\begin{align*}
\left|
\begin{matrix}
-1 & 2 & -2 \\
2 & 0 & 1 \\
3 & 1 & 2
\end{matrix}
\right|
&= 
-2 \left|
\begin{matrix}
2 & -2\\
1 & 2
\end{matrix}
\right| 
+
0 \left|
\begin{matrix}
-1 & -2 \\
3 & 2
\end{matrix}
\right|
-1 \left|
\begin{matrix}
-1 & 2 \\
3 & 1
\end{matrix}
\right| \\
&=
-2 (2 \times 2 - 1 \times (-2) ) + 0 -1 ((-1) \times 1 - 2 \times 3) \\
&= -12 +7 = -5
\end{align*}

\section{Properties of the determinant §5.3}
\begin{enumerate}[ (a) ]
\item $A$ is invertible $\Leftrightarrow \text{det}(A) \not = 0$.
\item $\text{det}(AB)=\text{det}(A)\text{det}(B)$
\item $\text{det}(\mathbb{I}_n) = 1$
\item $\text{det}(A^T) = \text{det}(A)$
\end{enumerate}

\section{Inverting matrices using determinants §5.3, 5.4}
Given an $n \times n$ matrix $A = [ A_{ij} ]$, let
\[ C_{ij}, \quad 1 \leq i,\ j \leq n \quad (C_{ij} = (-1)^{i+j} |\tilde{M}_{ij}| )\]
denote all the cofactors of $A$. Then the \textbf{adjugate matrix} of $A$ is defined by
\[ \text{adj}A = [C_{ij}]^T \]
If $A$ is non-singular (i.e. $\text{det}(A) \not = 0$) then
\[ A^{-1} = \frac{1}{\text{det}(A)} \text{adj}A \]

\subsection{Example}
Compute the inverse of
\[
A = 
\begin{bmatrix}
-1 & 2 & -2 \\
2 & 0 & 1 \\
3 & 1 & 2
\end{bmatrix}
\]

\subsection{Solution}
\[ 
M_{11} = \left| \begin{matrix}
0 & 1 \\
1 & 2
\end{matrix} \right|
= - 1,
M_{12} = \left| \begin{matrix}
2 & 1 \\
3 & 2
\end{matrix} \right|
= 1,
M_{13} = \left| \begin{matrix}
2 & 0 \\
3 & 1
\end{matrix} \right|
= 2
\]
\[
M_{21} = \left| \begin{matrix}
2 & -2 \\
1 & 2
\end{matrix} \right|
= 6,
M_{22} = \left| \begin{matrix}
-1 & -2 \\
3 & 2
\end{matrix} \right|
= 4,
M_{23} = \left| \begin{matrix}
-1 & 2 \\
3 & 1
\end{matrix} \right|
= -7
\]

\[
M_{31} = \left| \begin{matrix}
2 & -2 \\
0 & 1
\end{matrix} \right|
= 2,
M_{32} = \left| \begin{matrix}
-1 & -2 \\
2 & 1
\end{matrix} \right|
= 3,
M_{33} = \left| \begin{matrix}
-1 & 2 \\
2 & 0
\end{matrix} \right|
= -4
\]

\[
C = \begin{bmatrix}
+M_{11} & -M{12} & +M_{13} \\
-M_{21} & +M{22} & -M_{23} \\
+M_{31} & -M{32} & +M_{33}
\end{bmatrix}
= \begin{bmatrix}
-1 & -1 & 2 \\
-6 & 4 & 7 \\
2 & -3 & -4
\end{bmatrix}
\]

So then
\[ A^{-1} = \frac{1}{\text{det}(A)} \text{adj}A = \frac{1}{-5}C^T = -\frac{1}{5} \begin{bmatrix}
-1 & -6 & 2 \\
-1 & 4 & -3 \\
2 & 7 & -4
\end{bmatrix}  \]

\section{Cross product}
\begin{align*}
\vec{u} \times \vec{v} :&= \left| \begin{matrix}
\ihat & \jhat & \khat \\
u_1 & u_2 & u_3 \\
v_1 & v_2 & v_3
\end{matrix} \right| \\
&= \left| \begin{matrix}
u_2 & u_3 \\
v_2 & v_3
\end{matrix} \right| \ihat
- \left| \begin{matrix}
u_1 & u_3 \\
v_1 & v_3
\end{matrix} \right| \jhat
+ \left| \begin{matrix}
u_1 & u_2 \\
v_1 & v_2
\end{matrix} \right| \khat \\
&= (u_2v_3 - v_2u_3)\ihat - (u_1v_3 - v_1u_3)\jhat + (u_1v_2 - v_1u_2)\khat \\
&= (u_2v_3 - v_2u_3, u_1v_3 - v_1u_3, u_1v_2 - v_1u_2)
\end{align*}

\section{Solution space for $n \times n$ linear systems of equations}
\subsection{Theorem}
Suppose $A$ is an $n \times n$ matrix and $\vec{b}$ is an $n$-column vector.
\begin{enumerate}[ (a) ]
\item If $\text{det}(A) \not = 0$, then the linear system of equations
\[ A \vec{x} = \vec{b} \]
has a unique solution given by
\[ \vec{x} = A^{-1} \vec{b} \]
\item If $\vec{b} = \vec{0}$, then
\[ A \vec{x} = \vec{0} \]
has a non-trivial solution if and only if $\text{det}(A) = 0$.
\item If $\vec{b} \not = 0$ and $\text{det}(A) = 0$, then
\[ A \vec{x} = \vec{b} \]
can either have no solution or an infinite number of solutions. (You would have to compare the ranks of $A$ and $[A|\vec{b}]$ to determine which is true).
\end{enumerate}

\section{Eigenvalues and eigenvectors §5.7.2}
\subsection{Definition}
Given an $n \times n$ matrix $A$, a non-zero $n$-column vector $\vec{v}$ that satisfies
\[ A \vec{v} = \lambda \vec{v} \]
for some $\lambda \in \reals$ is called an \textbf{eigenvector}. The number $\lambda$ is called the real \textbf{eigenvalue} associated to $\vec{v}$.

Even though we will only consider real matrices $A = [A_{ij}],\ A_{ij} \in \reals$, it will be useful to allow for complex eigenvalues and eigenvectors. That is,
\[ A \vec{v} =\lambda \vec{v} \]
where $\lambda \in \complex$ and $\vec{v} = [v_j],\ v_j \in \complex,\ 1 \leq j \leq n$.

\subsection{Example}
Let
\[ 
A = \begin{bmatrix}
2 & 1 \\
4 & 2
\end{bmatrix}
\text{ and }
\vec{v} = \begin{bmatrix}
1 \\
2
\end{bmatrix}.
\]
Then
\[ A \vec{v} = 
\begin{bmatrix}
2 & 1 \\
4 & 2
\end{bmatrix}
\begin{bmatrix}
1 \\
2
\end{bmatrix}
=
\begin{bmatrix}
4 \\
8
\end{bmatrix}
=
4 \begin{bmatrix}
1 \\
2
\end{bmatrix}
\]
\[ \Rightarrow A \vec{v} = 4 \vec{v} \]
Thus $\vec{v}$ is a (real) eigenvector of $A$ with (real) eigenvalue $\lambda = 4$.

\end{document}