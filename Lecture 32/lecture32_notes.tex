\documentclass[11pt]{article}

\usepackage{amsmath}
\usepackage{amsfonts}
\usepackage{amssymb}

% Give ourself extra space for text
\usepackage[left = 2.2cm, right = 2.2cm, top = 1.8cm, bottom = 2.8cm]{geometry}

% Allows us to easily change the numbering system used in things like \begin{enumerate}. https://ctan.org/tex-archive/macros/latex/contrib/enumitem/
\usepackage[shortlabels]{enumitem}

% Turns table of contents, \refs, etc. into hyperlinks
\usepackage{hyperref}

% To include image, just use \includegraphics[scale=•]{relative path to image}
\usepackage{graphicx}
\graphicspath{ {./images/} }

% Common sets
\newcommand{\integers}{\mathbb{Z}}
\newcommand{\naturals}{\mathbb{N}}
\newcommand{\reals}{\mathbb{R}}
\newcommand{\complex}{\mathbb{C}}

% Power set
\newcommand{\powerset}{\mathcal{P}}

% Identity matrix
\newcommand{\ident}{\mathbb{I}}

% Inverse hyperbolic functions
\DeclareMathOperator{\arcosh}{arcosh}
\DeclareMathOperator{\arsinh}{arsinh}
\DeclareMathOperator{\artanh}{artanh}

% I hat, J hat, K hat
\newcommand{\ihat}{\boldsymbol{\hat{\textbf{\i}}}}
\newcommand{\jhat}{\boldsymbol{\hat{\textbf{\j}}}}
\newcommand{\khat}{\boldsymbol{\hat{\textbf{k}}}}

% Purely for Laplace symbol
\usepackage[scr]{rsfso}
\newcommand{\Laplace}{\mathscr{L}}

% Better vectors (for single characters)
\renewcommand{\vec}[1]{\mathbf{#1}}

% Allows us to number equations in \begin{align} statements, etc.
\newcommand\numberthis{\addtocounter{equation}{1}\tag{\theequation}}

% Augmented matrices: this allows us to make augmented matrics using something like \begin{bmatrix}[cc|c]. Taken from Stefan Kottwitz at https://tex.stackexchange.com/questions/2233/whats-the-best-way-make-an-augmented-coefficient-matrix.
\makeatletter
\renewcommand*\env@matrix[1][*\c@MaxMatrixCols c]{%
  \hskip -\arraycolsep
  \let\@ifnextchar\new@ifnextchar
  \array{#1}}
\makeatother

% NOTE: This means \section does NOT number sections, but ensures that they appear in the table of contents, which does not occur if simply \section* is used. From egreg @ https://tex.stackexchange.com/a/30225.
\setcounter{secnumdepth}{0} % sections are level 1

\begin{document}
\title{ENG1005: Lecture 32}
\author{Lex Gallon}
\maketitle

\tableofcontents

\section*{Video link}
\url{https://echo360.org.au/lesson/G_32340f5d-ff38-43d2-be9d-d88ddb1b3611_b944cecf-8ba5-40d3-a870-0243a0a9e78c_2020-06-03T14:58:00.000_2020-06-03T15:53:00.000/classroom#sortDirection=desc}

\section{Simple Laplace Transforms - continued}
\begin{align*}
\Laplace [1](s) &= \frac{1}{s},\ \text{Re}(s)>0 \\
\Laplace [t](s) &= \frac{1}{s^2},\ \text{Re}(s)>0 \\
\Laplace [e^{kt}](s) &= \frac{1}{s-k},\ \text{Re}(s)>\text{Re}(k)
\end{align*}

Applying the Laplace transform to
\[ \cos(\omega t) = \frac{e^{i \omega t} + e^{-i\omega t}}{2},\ \omega \in \reals, t \geq 0, \]
we see that
\begin{align*}
\Laplace[\cos(\omega t)](s) &= \frac{1}{2}\left( \Laplace[e^{i \omega t}](s) + \Laplace[e^{-i \omega t}](s) \right) \\
&= \frac{1}{2} \left( \frac{1}{s - i\omega} + \frac{1}{s + i\omega} \right),\ \text{Re}(s) > 0,
\end{align*}
or equivalently
\[ \Laplace[\cos(\omega t)](s) = \frac{s}{s^2 + \omega^2},\ \text{Re}(s) > 0. \]
Similar calculations starting from
\[ \sin(\omega t) = \frac{e^{i \omega t} - e^{-i\omega t}}{2} \]
shows that
\[ \Laplace[\sin(\omega t)](s) = \frac{\omega}{s^2 + \omega^2},\ \text{Re}(s) > 0. \]

\section{The s-shifting property §11.2.4}
\subsection{Theorem}
Suppose $f(t),\ t \geq 0$, is a function whose Laplace transform
\[ F(s) = \Laplace[f(t)](s) \]
is defined for $\text{Re}(s)>\sigma$, and $a \in \complex$. Then
\[ \Laplace[e^{at} f(t)](s) = F(s - a),\ \text{Re}(s) > \sigma + \text{Re}(a). \]

\subsection{Proof}
From
\begin{align*}
\Laplace[e^{at}f(t)](s) &= \int_0^\infty e^{-st}e^{at}f(t)\, dt \\
&= \int_0^\infty e^{-(s-a)t}f(t)\, dt \\
&= F(s - a)
\text{where } F(s) &= \int_0^\infty e^{-st} f(t)\, dt,\ \text{Re}(s) > 0.
\end{align*}
But
\[ \text{Re}(s-a) > \sigma \Leftrightarrow \text{Re}(s) > \sigma + \text{Re}(a), \]
so we conclude that
\[ \Laplace[e^{at} f(t)](s) = F(s-a),\ \text{Re}(s) > \sigma + \text{Re}(a). \]

\subsection{Example}
Compute
\[ \Laplace[te^{2t}](s). \]

\subsection*{Solution}
Let
\[ F(s) = \Laplace[t] = \frac{1}{s^2}, \text{Re}(s) > 0. \]
So by the s-shifting property, we see that
\[ \Laplace[e^{2t}t](s) = F(s-2) = \frac{1}{(s-2)^2},\ \text{Re}(s) > 2. \]

\section{Derivatives of Laplace Transforms §11.2.4}
\subsection{Theorem}
Suppose $f(t)$, $t \geq 0$, is a function whose Laplace Transform
\[ F(s) = \Laplace[f(t)](s) \]
is defined for $\text{Re}(s) > \sigma$.\\
Then
\[ \Laplace[t^nf(t)] = (-1)^n \frac{d^nF(s)}{ds^n},\ \text{Re}(s) > 0,\ n \in \naturals. \]

\subsection{Proof}
Differentiating
\[ F(s) = \Laplace[f(t)](s) = \int_0^\infty e^{-st}f(t)\, dt,\ \text{Re}(s) > \sigma, \]
with respect to $s$ gives
\begin{align*}
\frac{d^n F(s)}{ds^n} &= \frac{d}{ds^n} \int_0^\infty e^{-st}f(t)\, dt \\
&= \int_0^\infty \frac{d}{ds^n} \left( e^{-st}f(t) \right) \, dt \\
&= \int_0^\infty \frac{d}{ds^n} \left( e^{-st} \right) f(t)  \, dt \\
&= \int_0^\infty (-1)^n t^n  e^{-st} f(t)  \, dt \\
&= (-1)^n \int_0^\infty e^{-st} (-1)^n f(t)  \, dt \\
&= (-1)^n \Laplace[t^n f(t)](s) \\
\Rightarrow \Laplace[t^n f(t)](s) &= (-1)^n \frac{d^n F(s)}{ds^n},\ \text{Re}(s) > \sigma.
\end{align*}
Q.E.D. (note we don't know how to prove that the domain remains the same but know that it is).

\subsection{Example}
Compute
\[ \Laplace[t^n](s). \]

\subsection*{Proof}
Let
\[ F(s) = \Laplace[1](s) = \frac{1}{s},\ \text{Re}(s) > 0. \]
Then
\[ \Laplace[t^n](s) = \Laplace[t^n \cdot 1](s) = (-1)^n \frac{d^nF(s)}{ds^n} = (-1)^n \frac{d^n}{ds^n} \left( \frac{1}{s} \right). \]
It's easy to see that
\[ \Laplace[t^n](s) = (-1)^n \frac{d^n}{ds^n}\left( \frac{1}{s} \right) = \frac{n!}{s^{n+1}}. \]
So in summary
\[ \Laplace[t^n](s) = \frac{n!}{s^{n+1}},\ \text{Re}(s) > 0. \]

\section{The Laplace Transforms of derivatives §11.3.1}
\subsection{Theorem}
Suppose that $f(t),\ t \geq 0$ satisfies 
\[ \left| \frac{d^jf}{dt^j}(t) \right| \leq Me^{\sigma t},\ h=1,2,...,n \]
for some $M > 0$, and $\sigma \in \reals$. Then,
\[ \Laplace \left[ \frac{d^jf}{dt^j}(t) \right](s) = s^n F(s) - \sum_{j=1}^n s^{n-j}\frac{d^{j-1}f}{dt^{j-1}}(0),\ j=1, 2,..., n \]
where
\[ F(s) = \Laplace[f(t)](s). \]
Some examples:
\begin{align*}
\Laplace \left[ \frac{df}{dt}(t) \right](s) &= s \Laplace[f(t)](s) - f(0) \\
\Laplace \left[ \frac{d^2f}{dt^2}(t) \right](s) &= s^2 \Laplace[f(t)](s) - s f(0) - \frac{df}{dt}(0)
\end{align*}

\subsection{Proof}
From the definition
\[ \Laplace \left[ \frac{df}{dt}(t) \right](s) = \int_0^\infty e^{-st} \frac{df}{dt}(t) \, dt, \]
we see that
\begin{align*}
\Laplace \left[ \frac{df}{dt}(t) \right](s) &= \int_0^\infty e^{-st} \frac{df}{dt}(t) \, dt \\
&= \lim_{T \rightarrow \infty} \int_0^T e^{-st} \frac{df}{dt}(t) \, dt \\
&= \lim_{T \rightarrow \infty} \left[ \left. e^{-st}f(t) \right|_0^T - \int_0^T \frac{d}{dt} \left( e^{-st} \right) f(t) \, dt \right] \\
&= \lim_{T \rightarrow \infty} \left[ e^{-sT}f(T) - f(0) + s\int_0^T e^{-st} f(t) \, dt \right] \\
\end{align*}
But
\begin{align*}
\left| \lim_{T \rightarrow \infty} e^{-sT}f(T) \right| &= \lim_{T \rightarrow \infty} \left|  e^{-sT}f(T) \right| \\
&= \lim_{T \rightarrow \infty} \left| e^{-\text{Re}(s)T} e^{-\text{Im}(s)iT} f(T) \right| \\
&= \lim_{T \rightarrow \infty} \left| e^{-\text{Re}(s)T} \right| \left| e^{-\text{Im}(s)iT} \right| \left| f(T) \right| \\
&= \lim_{T \rightarrow \infty} e^{-\text{Re}(s)T} \left| f(T) \right| \\
&= \lim_{T \rightarrow \infty} e^{-\text{Re}(s)T} Me^{\sigma T} \\
&= M \lim_{T \rightarrow \infty} e^{-(\text{Re}(s) - \sigma)T} = 0,\ \text{Re}(s) > \sigma. \\
\end{align*}
Thus,
\begin{align*}
\Laplace \left[ \frac{df}{dt}(t) \right](s) &= -f(0) + s\Laplace[f(t)](s)
\end{align*}

\end{document}