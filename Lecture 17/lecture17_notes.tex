\documentclass[11pt]{article}

\usepackage{amsmath}
\usepackage{amsfonts}
\usepackage{amssymb}

% Give ourself extra space for text
\usepackage[left = 2.2cm, right = 2.2cm, top = 1.8cm, bottom = 2.8cm]{geometry}

% Allows us to easily change the numbering system used in things like \begin{enumerate}. https://ctan.org/tex-archive/macros/latex/contrib/enumitem/
\usepackage[inline, shortlabels]{enumitem}

% Turns table of contents, \refs, etc. into hyperlinks
\usepackage{hyperref}

% To include image, just use \includegraphics[scale=•]{relative path to image}
\usepackage{graphicx}
\graphicspath{ {./images/} }

% Common sets
\newcommand{\integers}{\mathbb{Z}}
\newcommand{\naturals}{\mathbb{N}}
\newcommand{\reals}{\mathbb{R}}

% Inverse hyperbolic functions
\DeclareMathOperator{\arcosh}{arcosh}
\DeclareMathOperator{\arsinh}{arsinh}
\DeclareMathOperator{\artanh}{artanh}

% I hat, J hat, K hat
\newcommand{\ihat}{\boldsymbol{\hat{\textbf{\i}}}}
\newcommand{\jhat}{\boldsymbol{\hat{\textbf{\j}}}}
\newcommand{\khat}{\boldsymbol{\hat{\textbf{k}}}}

% Better vectors (for single characters)
\renewcommand{\vec}[1]{\mathbf{#1}}

% Allows us to number equations in \begin{align} statements, etc.
\newcommand\numberthis{\addtocounter{equation}{1}\tag{\theequation}}

% Augmented matrices: this allows us to make augmented matrics using something like \begin{bmatrix}[cc|c]. Taken from Stefan Kottwitz at https://tex.stackexchange.com/questions/2233/whats-the-best-way-make-an-augmented-coefficient-matrix.
\makeatletter
\renewcommand*\env@matrix[1][*\c@MaxMatrixCols c]{%
  \hskip -\arraycolsep
  \let\@ifnextchar\new@ifnextchar
  \array{#1}}
\makeatother

% NOTE: This means \section does NOT number sections, but ensures that they appear in the table of contents, which does not occur if simply \section* is used. From egreg @ https://tex.stackexchange.com/a/30225.
\setcounter{secnumdepth}{0} % sections are level 1

\begin{document}
\title{ENG1005: Lecture 17}
\author{Lex Gallon}
\maketitle

\tableofcontents

\section*{Video link}
Click \href{https://echo360.org.au/lesson/G_32340f5d-ff38-43d2-be9d-d88ddb1b3611_b944cecf-8ba5-40d3-a870-0243a0a9e78c_2020-04-29T14:58:00.000_2020-04-29T15:53:00.000/classroom}{here} for a recording of the lecture.

\section{Matrix inverses - continued}
$A$ is an invertible matrix (non-singular) if there exists a matrix $A^{-1}$ such that
\[ AA^{-1} = A^{-1}A = \mathbb{I}_n \]

\subsection{2x2 matrix inversion}
Given a $2 \times 2$ matrix
\[ A = \begin{bmatrix}
a & b \\
c & d
\end{bmatrix} \]
then
\[ A^{-1} = \frac{1}{ad -bc} \begin{bmatrix}
d & -b\\
-c & a
\end{bmatrix} \]
provided
\[ ad - bc \not = 0 \text{ (i.e. } A \text{ is invertible)} \]

\subsection{Example}
Solve
\begin{align*}
x+ 2y &= 1, \\
x + 4y &= -1
\end{align*}

\subsection{Solution}
This can be repesented as
\[
\begin{bmatrix}
1 & 2 \\
1 & 4
\end{bmatrix}
\begin{bmatrix}
x\\
y
\end{bmatrix}
= \begin{bmatrix}
1\\
-1
\end{bmatrix}
\]

So,
\begin{align*}
\begin{bmatrix}
x\\
y
\end{bmatrix}
&=
\begin{bmatrix}
1 & 2 \\
1 & 4
\end{bmatrix}^{-1}
\begin{bmatrix}
1\\
-1
\end{bmatrix} \\
&= \frac{1}{1\times 4 - 1\times 2} \begin{bmatrix}
4 & -2 \\
-1 & 1
\end{bmatrix}
\begin{bmatrix}
1\\
-1
\end{bmatrix} \\
&= \frac{1}{2}
\begin{bmatrix}
6 \\
-2 \\
\end{bmatrix} \\
&= \begin{bmatrix}
3\\
-1
\end{bmatrix}
\end{align*}
So $x=3$, $y=-1$.

\section{Matrix inversion via Gauss-Jordan elimination}
\subsection{Matrix inversion algorithm}
\begin{enumerate}[ (i) ]
\item Given an $n \times n$ matrix $A$, form the augmented matrix
\[ [A | \mathbb{I}_n ] \]

E.g.\\
If
\[ A = \begin{bmatrix}
4 & -7\\
3 & 2
\end{bmatrix}
, \text{ then } [ A | \mathbb{I}_n ]
= \begin{bmatrix}[cc|cc]
4 & -7 & 1 & 0 \\
3 & 2 & 0 & 1
\end{bmatrix}
\]

\item Perform elementary row operations on $[A | \mathbb{I}_n ]$ until one of the following happens:
\begin{enumerate}[ label=(\roman{enumi}.\alph*) ]
\item 
$[A | \mathbb{I}_n ]$ is transformed into $[\mathbb{I}_n | B ]$. In this case, $A$ is non-singular with inverse $A^{-1} = B$.
\item $[A | \mathbb{I}_n ]$ is transformed into $[\tilde{A} | \tilde{B} ]$, where $\tilde{A}$ has a row of zeros. In this case, $A$ is singular.
\end{enumerate}

\end{enumerate}

\subsection{Example}
Determine if the following matrices are invertible:
\begin{enumerate}[ (a) ]
\item
\[
\begin{bmatrix}
1 & 2 & 3\\
-1 & -1 & -1\\
0 & 2 & 4
\end{bmatrix}
\]
\item
\[
\begin{bmatrix}
1 & 2 & 1 \\
-1 & -1 & -1\\
0 & 2 & 3
\end{bmatrix}
\]
\end{enumerate}

\subsection{Solution}
\begin{enumerate}[ (a) ]
\item 
\begin{align*}
\begin{bmatrix}
1 & 2 & 3\\
-1 & -1 & -1\\
0 & 2 & 4
\end{bmatrix}
&\rightarrow
\begin{bmatrix}[ccc|ccc]
1 & 2 & 3 & 1 & 0 & 0\\
-1 & -1 & -1 & 0 & 1 & 0\\
0 & 2 & 4 & 0 & 0 & 1\\
\end{bmatrix} \\
&\rightarrow
\begin{bmatrix}[ccc|ccc]
1 & 2 & 3 & 1 & 0 & 0\\
0 & 1 & 2 & 1 & 1 & 0\\
0 & 2 & 4 & 0 & 0 & 1\\
\end{bmatrix} 
\begin{matrix}
\\
R_2 \rightarrow R_2 + R_1 \\
\\
\end{matrix} \\
&\rightarrow
\begin{bmatrix}[ccc|ccc]
1 & 2 & 3 & 1 & 0 & 0\\
0 & 1 & 2 & 1 & 1 & 0\\
0 & 0 & 0 & -2 & -2 & 1
\end{bmatrix}
\begin{matrix}
\\
\\
R_3 \rightarrow R_3 - 2R_2
\end{matrix} \\
\end{align*}

But notice that we now have a row of zeros on the left half of the augmented matrix. This shows that the given matrix
\[
\begin{bmatrix}
1 & 2 & 3\\
-1 & -1 & -1\\
0 & 2 & 4
\end{bmatrix}
\]
is singular/not invertible.

\item
\begin{align*}
\begin{bmatrix}
1 & 2 & 1 \\
-1 & -1 & -1\\
0 & 2 & 3
\end{bmatrix}
&\rightarrow
\begin{bmatrix}[ccc|ccc]
1 & 2 & 1 & 1 & 0 & 0\\
-1 & -1 & -1 & 0 & 1 & 0\\
0 & 2 & 3 & 0 & 0 & 1
\end{bmatrix} \\
&\rightarrow
\begin{bmatrix}[ccc|ccc]
1 & 2 & 1 & 1 & 0 & 0\\
0 & 1 & 0 & 1 & 1 & 0\\
0 & 2 & 3 & 0 & 0 & 1
\end{bmatrix}
\begin{matrix}
\\
R_2 \rightarrow R_2 + R_1 \\
\\
\end{matrix} \\
&\rightarrow
\begin{bmatrix}[ccc|ccc]
1 & 2 & 1 & 1 & 0 & 0\\
0 & 1 & 0 & 1 & 1 & 0\\
0 & 0 & 3 & -2 & -2 & 1
\end{bmatrix}
\begin{matrix}
\\
\\
R_3 \rightarrow R_3 - 2R_2
\end{matrix} \\
&\rightarrow
\begin{bmatrix}[ccc|ccc]
1 & 2 & 1 & 1 & 0 & 0\\
0 & 1 & 0 & 1 & 1 & 0\\
0 & 0 & 1 & -\frac{2}{3} & -\frac{2}{3} & \frac{1}{3}
\end{bmatrix}
\begin{matrix}
\\
\\
R_3 \rightarrow \frac{1}{3}R_3
\end{matrix} \\
&\rightarrow
\begin{bmatrix}[ccc|ccc]
1 & 0 & 1 & -1 & -2 & 0\\
0 & 1 & 0 & 1 & 1 & 0\\
0 & 0 & 1 & -\frac{2}{3} & -\frac{2}{3} & \frac{1}{3}
\end{bmatrix}
\begin{matrix}
R_1 \rightarrow R_1 - 2R_2 \\
\\
\\
\end{matrix} \\
&\rightarrow
\begin{bmatrix}[ccc|ccc]
1 & 0 & 0 & -\frac{1}{3} & -\frac{4}{3} & -\frac{1}{3} \\
0 & 1 & 0 & 1 & 1 & 0\\
0 & 0 & 1 & -\frac{2}{3} & -\frac{2}{3} & \frac{1}{3}
\end{bmatrix}
\begin{matrix}
R_1 \rightarrow R_1 - R_3 \\
\\
\\
\end{matrix}
\end{align*}

Now we have the identity matrix on the left half of the augmented matrix. This means that
\[
\begin{bmatrix}
1 & 2 & 1 \\
-1 & -1 & -1\\
0 & 2 & 3
\end{bmatrix}^{-1}
=
\begin{bmatrix}
-\frac{1}{3} & -\frac{4}{3} & -\frac{1}{3} \\
1 & 1 & 0\\
-\frac{2}{3} & -\frac{2}{3} & \frac{1}{3}
\end{bmatrix}
\]
\end{enumerate}

\section{Determinants §5.3}
\subsection{2x2 matrices}
The determinant of a 2x2 matrix $\displaystyle{A = \begin{bmatrix}
a & b\\
c & d
\end{bmatrix}}$ is defined by
\[ \text{det}(A) = ad - bc (= |A|) \]
Recalling that
\[ A^{-1} = \frac{1}{\text{det}(A)} \begin{bmatrix}
d & -b\\
-c & a
\end{bmatrix},
\quad \text{det}(A) = ad -bc \]
we see that $A$ is invertible if and only if $\text{det}(A) \not = 0$.

\subsection{$n$x$n$ matrices}
If $A= [A_{ij}]$ is an $n \times n$ matrix, let $\tilde{M}_{ij}$ be the $(n-1) \times (n-1)$ matrix defined by deleting the $i$th row and $j$th column from the matrix $A$.
E.g. \\
If $\displaystyle{A = \begin{bmatrix}
1 & 3 & 2\\
-1 & 1 & 4\\
0 & 2 & 5
\end{bmatrix}}$, then $\displaystyle{\tilde{M}_{12} =  \begin{bmatrix}
-1 & 4 \\
0 & 5
\end{bmatrix}  }$
The numbers
\[ M_{ij} = \text{det}(\tilde{M}_{ij}) \]
are known as minors.

The determinant of the $n \times n$ matrix $A=[A_{ij}]$ is then defined recursively as
\[ \text{det}(A) = \sum_{j=1}^n (-1)^{i+j}A_{ij} M_{ij} \]
where $i$ is any fixed number on $\{ 1, 2, ..., n \}$. I.e. you get to choose $i$ (same result no matter what)!

The numbers
\[ C_{ij} = (-1)^{i+j}M_{ij} \]
are called cofactors. So if you like, you can rewrite the determinant formula as
\[ \text{det}(A) = \sum_{j=1}^n A_{ij} C_{ij},\quad i \in \{1, 2, ..., n\} \]

This formula is known as the Laplace or cofactor expansion for the determinant of $A$.

Note: The number $(-1)^{i+j}$ can be remembered by
\[
\begin{bmatrix}
+ & - & + & ... \\
- & + & - & ... \\
\vdots
\end{bmatrix}
\]

\end{document}