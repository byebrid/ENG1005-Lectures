\documentclass[11pt]{article}

\usepackage{amsmath}
\usepackage{amsfonts}
\usepackage{amssymb}

% Purely for Laplace symbol
\usepackage[scr]{rsfso}
\newcommand{\Laplace}{\mathscr{L}}

% Give ourself extra space for text
\usepackage[left = 2.2cm, right = 2.2cm, top = 1.8cm, bottom = 2.8cm]{geometry}

% Allows us to easily change the numbering system used in things like \begin{enumerate}. https://ctan.org/tex-archive/macros/latex/contrib/enumitem/
\usepackage[shortlabels]{enumitem}

% Turns table of contents, \refs, etc. into hyperlinks
\usepackage{hyperref}

% To include image, just use \includegraphics[scale=•]{relative path to image}
\usepackage{graphicx}
\graphicspath{ {./images/} }

% Common sets
\newcommand{\integers}{\mathbb{Z}}
\newcommand{\naturals}{\mathbb{N}}
\newcommand{\reals}{\mathbb{R}}
\newcommand{\complex}{\mathbb{C}}

% Power set
\newcommand{\powerset}{\mathcal{P}}

% Identity matrix
\newcommand{\ident}{\mathbb{I}}

% Inverse hyperbolic functions
\DeclareMathOperator{\arcosh}{arcosh}
\DeclareMathOperator{\arsinh}{arsinh}
\DeclareMathOperator{\artanh}{artanh}

% I hat, J hat, K hat
\newcommand{\ihat}{\boldsymbol{\hat{\textbf{\i}}}}
\newcommand{\jhat}{\boldsymbol{\hat{\textbf{\j}}}}
\newcommand{\khat}{\boldsymbol{\hat{\textbf{k}}}}

% Better vectors (for single characters)
\renewcommand{\vec}[1]{\mathbf{#1}}

% Allows us to number equations in \begin{align} statements, etc.
\newcommand\numberthis{\addtocounter{equation}{1}\tag{\theequation}}

% Augmented matrices: this allows us to make augmented matrics using something like \begin{bmatrix}[cc|c]. Taken from Stefan Kottwitz at https://tex.stackexchange.com/questions/2233/whats-the-best-way-make-an-augmented-coefficient-matrix.
\makeatletter
\renewcommand*\env@matrix[1][*\c@MaxMatrixCols c]{%
  \hskip -\arraycolsep
  \let\@ifnextchar\new@ifnextchar
  \array{#1}}
\makeatother

% NOTE: This means \section does NOT number sections, but ensures that they appear in the table of contents, which does not occur if simply \section* is used. From egreg @ https://tex.stackexchange.com/a/30225.
\setcounter{secnumdepth}{0} % sections are level 1

\begin{document}
\title{ENG1005: Lecture 31}
\author{Lex Gallon}
\maketitle

\tableofcontents

\section*{Video link}
\url{https://echo360.org.au/lesson/G_8402119b-734b-4e1e-a3b4-7e907e86ddba_b944cecf-8ba5-40d3-a870-0243a0a9e78c_2020-06-02T15:58:00.000_2020-06-02T16:53:00.000/classroom#sortDirection=desc}

\section{Laplace Transform §11.2, 11.2.1}
Given a function $f(t)$ defined for $t \geq 0$, the Laplace Transform of $f$ is defined by
\[ F(s) = \int_0^\infty e^{-st}f(t)\, dt,\quad s \in D \subset \complex, \]
where $D$ is some sort of domain. Note you could consider $F$ as a function of two variables, namely the real and imaginary components of $s$, but I guess we'll see defining it as above is preferable.

The function $F(s)$ is often written
\[ F(s) = \Laplace[f](s). \]

It can be used to transform something from a time domain ($f(t)$)into a complex frequency domain ($\Laplace[f](s)$).

\section{Linearity §11.2.4}
Given $\lambda \in \complex$, we have
\begin{align*}
\Laplace[\lambda f_1(t) +f_2(t)](s) &= \int_0^\infty e^{-ts}(\lambda f_1(t) +f_2(t))\, dt \\
&= \lambda \int_0^\infty e^{-ts}f_1(t)\, dt + \int_0^\infty e^{-st}f_2(t)\, dt \\
&= \lambda \Laplace[f_1(t)](s) + \Laplace[f_2(t)](s) \\
\Rightarrow \Laplace[\lambda f_1(t) +f_2(t)](s) &= \lambda \Laplace[f_1(t)](s) + \Laplace[f_2(t)](s)
\end{align*}
for all $\lambda \in \complex$ and $f_1(t)$, $f_2(t)$ defined for $t\geq 0$.

\section{Laplace Transforms of simple functions §11.2.2}
\subsection{Example}
Let
\[ f(t) = 1,\ t \geq 0. \]
Then
\begin{align*}
\Laplace[1](s) &= \int_0^\infty e^{-st}\, dt \\
&= \lim_{T \rightarrow \infty} \int_0^T e^{-st}\, dt \\
&= \lim_{T \rightarrow \infty} \left[\left. -1\frac{1}{s}e^{-st} \right|_0^T \right] \\
&= \lim_{T \rightarrow \infty} \left[-\frac{1}{s}e^{-sT} + \frac{1}{s} \right] \\
&= \lim_{T \rightarrow \infty}-\frac{1}{s}e^{-sT} + \frac{1}{s} \\
&= -\frac{1}{s} \lim_{T \rightarrow \infty} e^{-\theta T - i\omega T} + \frac{1}{s} \\
&= -\frac{1}{s} \lim_{T \rightarrow \infty} e^{-\sigma T} e^{-i\omega T} + \frac{1}{s} \\
&= -\frac{1}{s} \lim_{T \rightarrow \infty} e^{-\sigma T} (\cos(\omega T) - i\sin(\omega T)) + \frac{1}{s}.
\end{align*}
Note the magnitude
\[ \left| e^{-i\omega T} \right| = |\cos(\omega T) - i\sin(\omega T)| = 1. \]
So we now have
\begin{align*}
\Laplace[1](s) &= \frac{1}{s} \quad \text{ IF } \sigma = \text{Re}(s) > 0.
\end{align*}
We got this by noting that, for that integral to converge, that limit had to reach zero, meaning
\[ \lim_{T \rightarrow \infty} e^{-\sigma T} = 0, \]
which will only happen for positive $\sigma$.

\subsection{Example}
Let
\[ f(t) = t,\ t\geq 0. \]
Then
\begin{align*}
\Laplace[t](s) &= \int_0^\infty e^{-st} t \, dt \\
&= \lim_{T \rightarrow \infty} \int_0^T e^{-st} t \, dt \\
&= \lim_{T \rightarrow \infty} \left[ \left. -\frac{e^{-st}}{s}t \right|_0^T + \frac{1}{s} \int_0^T e^{-st}\, dt \right] \\
&= \lim_{T \rightarrow \infty} -\frac{e^{-sT}}{s}T + \frac{1}{s}\lim_{T \rightarrow \infty} \int_0^T e^{-st}\, dt \\
&=\lim_{T \rightarrow \infty} -\frac{e^{-sT}}{s}T + \frac{1}{s} \Laplace[1](s),\ \text{Re}(s) > 0 \\
&= \lim_{T \rightarrow \infty} -\frac{e^{-sT}}{s}T + \frac{1}{s^2} \\
&= 0 + \frac{1}{s^2} \\
&= \frac{1}{s^2}.
\end{align*}
Thus,
\[ \Laplace[t](s) = \frac{1}{s^2},\ \text{Re}(s) > 0. \]

\section{Example}
Let
\[ f(t) = e^{kt},\ k \in \complex,\ t \geq 0. \]
Then
\begin{align*}
\Laplace[e^{kt}](s) &= \int_0^\infty e^{-st}e^{kt}\, dt \\
&= \int_0^\infty e^{t(k-s)}\, dt \\
&= \lim_{T \rightarrow \infty} \int_0^T e^{t(k-s)}\, dt \\
&= \lim_{T \rightarrow \infty} \left[ \left. \frac{1}{k-s}e^{t(k-s)} \right|_0^T \right] \\
&= \lim_{T \rightarrow \infty} \left[ \frac{1}{k-s}e^{T(k-s)} - \frac{1}{k-s} \right] \\
&= \frac{1}{k-s} \lim_{T \rightarrow \infty} e^{-(s-k)T} + \frac{1}{s-k}.
\end{align*}
We now want
\[ \text{Re}(s-k) > 0 \Rightarrow \text{Re}(s) > \text{Re}(k), \]
so the limit reaches zero, so we then get
\[ \Laplace[e^{kt}](s) = \frac{1}{s-k},\ \text{Re}(s) > \text{Re}(k) \]

\section{Existence of the Laplace Transform §11.2.3}
\subsection{Theorem}
If $f(t)$ is a piecewise continuous function for $t \geq 0$ and there exists constants $M \geq 0, \lambda \in \reals$ such that
\[ |f(t)| \leq Me^{\lambda t},\ t \geq 0, \] 
then the Laplace Transform
\[ \Laplace[f](s) = \int_0^\infty e^{-st}f(t)\, dt \]
exists and is well-defined for $\text{Re}(s) > \lambda$.

To see why this is the case, we note that
\begin{align*}
\left| \int_0^T e^{-st} f(t)\, dt \right| &\leq \int_0^T \left| e^{-st}f(t) \right|\, dt \\
&\leq \int_0^T |e^{-st}| |f(t)| \, dt.
\end{align*}
Letting
\[ s = \sigma + i \omega, \]
we see that
\[ e^{-st} = e^{-\sigma t + i\omega t} = e^{-\sigma t}e^{i \omega t} \]
from which we conclude that
\[ e^{-st} = |e^{-\sigma t}| |e^{i \omega t}| = |e^{-\sigma t}|. \]
Substituting this into the inequality gives
\[ \left| \int_0^T e^{-st} f(t)\, dt \right| \leq \int_0^T e^{-\sigma t} |f(t)| \, dt.  \]
Now, if
\[ |f(t)| \leq Me^{\lambda t}, \]
then
\begin{align*}
\left| \int_0^T e^{-st} f(t)\, dt \right| &\leq M \int_0^T e^{-\sigma t} e^{\lambda t} \, dt \\
&\leq  M \int_0^T e^{-(\sigma - \lambda) t} \, dt \\
&=\leq M \left[ \left. \frac{-1}{\sigma - \lambda} e^{-(\sigma - \lambda)t} \right|_0^T \right] \\
&\leq M \left[ \frac{-1}{\sigma - \lambda} e^{-(\sigma - \lambda)T} + \frac{1}{\sigma - \lambda} \right]
\end{align*}

Assuming that
\[ \sigma - \lambda >0 \Leftrightarrow \text{Re}(s) > \lambda, \]
we see from letting $T \rightarrow \infty$ in the above inequality that 
\[ \left| \int_0^T e^{-st} f(t)\, dt \right| \leq \frac{M}{\sigma - \lambda}, \]
and
\[ \int_0^\infty e^{-st} f(t)\, dt \text{ exists.} \]

\end{document}