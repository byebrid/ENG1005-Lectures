\documentclass[11pt]{article}

\usepackage{amsmath}
\usepackage{amsfonts}
\usepackage{amssymb}

% Give ourself extra space for text
\usepackage[left = 2.2cm, right = 2.2cm, top = 1.8cm, bottom = 2.8cm]{geometry}

% Allows us to easily change the numbering system used in things like \begin{enumerate}. https://ctan.org/tex-archive/macros/latex/contrib/enumitem/
\usepackage[shortlabels]{enumitem}

% Turns table of contents, \refs, etc. into hyperlinks
\usepackage{hyperref}

% To include image, just use \includegraphics[scale=•]{relative path to image}
\usepackage{graphicx}
\graphicspath{ {./images/} }

% Common sets
\newcommand{\integers}{\mathbb{Z}}
\newcommand{\naturals}{\mathbb{N}}
\newcommand{\reals}{\mathbb{R}}

% Power set
\newcommand{\powerset}{\mathcal{P}}

% Identity matrix
\newcommand{\ident}{\mathbb{I}}

% Inverse hyperbolic functions
\DeclareMathOperator{\arcosh}{arcosh}
\DeclareMathOperator{\arsinh}{arsinh}
\DeclareMathOperator{\artanh}{artanh}

% Rank
\DeclareMathOperator{\rank}{rank}

% I hat, J hat, K hat
\newcommand{\ihat}{\boldsymbol{\hat{\textbf{\i}}}}
\newcommand{\jhat}{\boldsymbol{\hat{\textbf{\j}}}}
\newcommand{\khat}{\boldsymbol{\hat{\textbf{k}}}}

% Better vectors (for single characters)
\renewcommand{\vec}[1]{\mathbf{#1}}

% Allows us to number equations in \begin{align} statements, etc.
\newcommand\numberthis{\addtocounter{equation}{1}\tag{\theequation}}

% Augmented matrices: this allows us to make augmented matrics using something like \begin{bmatrix}[cc|c]. Taken from Stefan Kottwitz at https://tex.stackexchange.com/questions/2233/whats-the-best-way-make-an-augmented-coefficient-matrix.
\makeatletter
\renewcommand*\env@matrix[1][*\c@MaxMatrixCols c]{%
  \hskip -\arraycolsep
  \let\@ifnextchar\new@ifnextchar
  \array{#1}}
\makeatother

% NOTE: This means \section does NOT number sections, but ensures that they appear in the table of contents, which does not occur if simply \section* is used. From egreg @ https://tex.stackexchange.com/a/30225.
\setcounter{secnumdepth}{0} % sections are level 1

\begin{document}
\title{ENG1005: Lecture 19}
\author{Lex Gallon}
\maketitle

\tableofcontents

\section*{Video link}
Click \href{https://echo360.org.au/lesson/G_8402119b-734b-4e1e-a3b4-7e907e86ddba_b944cecf-8ba5-40d3-a870-0243a0a9e78c_2020-05-05T15:58:00.000_2020-05-05T16:53:00.000/classroom#sortDirection=desc}{here} for a recording of the lecture.

\section{Eigenvalues - continued}
\subsection{Example}
\[ 
\text{Let } A = \begin{bmatrix}
0 & -1 \\
1 & 0
\end{bmatrix}
\text{ and }
\vec{v} = \begin{bmatrix}
-1 \\
i
\end{bmatrix}
\]

Then,
\[ 
A\vec{v} = 
\begin{bmatrix}
0 & -1 \\
1 & 0
\end{bmatrix}
\begin{bmatrix}
-1 \\
i
\end{bmatrix}
= \begin{bmatrix}
-i \\
-1
\end{bmatrix}
= \begin{bmatrix}
-i \\
i^2
\end{bmatrix}
= i \begin{bmatrix}
-1 \\
i
\end{bmatrix}
\]
\[ 
\Rightarrow A\vec{v} = i\vec{v}
\]
So $\vec{v}$ is a complex eigenvector of $A$ with eigenvalue $\lambda = i$.

\section{Characteristic equation}
If $\vec{v}$ is an eigenvector of matrix $A$, then
\[ A \vec{v} = \lambda \vec{c} \]
\[ (A - \lambda \mathbb{I})\vec{v} = \vec{0} \]
This equation can have a non-trivial solution if and only if $\det(A - \lambda \mathbb{I})=0$.
So, to find eigenvalues, we want to solve the equation
\[ \det(A - \lambda \ident)=0 \]
for $\lambda$. This is the \textbf{characteristic equation} of $A$.

\subsection{Remarks}
\begin{enumerate}[ (a) ]
\item If $\lambda$ satisfies the characteristic equation $\det(A - \lambda \ident)=0$, then there exists at least 1 eigenvector $\vec{v}$ of $A$ satisfying
\[ A \vec{v} = \lambda \vec{v} \]
\item If $\vec{v}$ is an eigenvector of $A$ with eigenvalue $\lambda$, then $\lambda$ satisfies the characteristic equation $\det(A - \lambda \ident)=0$.
\item When written out explicitly, we have
\[ \text{c}(\lambda) := \det(A - \lambda \ident _n) = c_0 + c_1 \lambda + ... + c_n \lambda^n \]
where $A$ is an $n \times n$ matrix. This polynomial is known as the \textbf{characteristic polynomial} of $A$. It can have, at most, $n$ distinct (complex) roots.
\item If $\vec{v}$ is an eigenvector of $A$, then so is $\alpha \vec{v}$ for any $\alpha \not = 0$.
\[ A(\alpha \vec{v}) = \alpha A \vec{v} = \alpha \lambda \vec{v} = \lambda (\alpha \vec{v}) \]
\[ \Rightarrow A(\alpha \vec{v}) = \lambda (\alpha \vec{v}) \]
\item If $\vec{v}$ is a complex eigenvector of a real matrix $A$ with a complex eigenvalue $\lambda$, then the complex conjugate $\overline{\vec{v}}$ is also a complex eigenvector of $A$ with a complex eigenvalue $\overline{\lambda}$.
\[ \vec{v} = \begin{bmatrix}
v_1 \\
v_2 \\
\vdots \\
v_n
\end{bmatrix} 
\Rightarrow \overline{\vec{v}} = \begin{bmatrix}
\overline{v_1} \\
\overline{v_2} \\
\vdots \\
\overline{v_n}
\end{bmatrix} 
\]

\[ 
A = \begin{bmatrix}
0 & -1 \\
1 & 0
\end{bmatrix},\ 
\vec{v} = \begin{bmatrix}
-1 \\
i
\end{bmatrix}
\Rightarrow A \vec{v} = \lambda \vec{v},\ \lambda = i
\]
\[ \Rightarrow A \overline{\vec{v}} = \overline{\lambda} \overline{\vec{v}},\ \overline{\lambda} = -i \]
\end{enumerate}

\subsection{Example}
Determine all the eigenvalues and eigenvectors of
\[ A = \begin{bmatrix}
2 & 1 \\
4 & 2
\end{bmatrix} \]

\subsection{Solution}
First, we compute the characteristic polynomial:
\[ 
\text{c}(\lambda) = \det(A - \lambda \ident) 
= \det(\begin{bmatrix}
2 & 1 \\
4 & 2
\end{bmatrix}
- \begin{bmatrix}
\lambda & 0 \\
0 & \lambda
\end{bmatrix} 
= \det(\begin{bmatrix}
2 - \lambda & 1 \\
4 & 2 - \lambda
\end{bmatrix})
= (2 - \lambda)^2 - 4
\]
\[
\Rightarrow \text{c}(\lambda) = \lambda^2 - 4\lambda = \lambda (\lambda - 4)
\]
Thus, $\text{c}(\lambda) = 0 \Leftrightarrow \lambda = 0 \text{ or } \lambda = 4$. So we now have the eigenvalues of $A$.

\underline{The eigenvector associated with $\lambda = 4$:}
\[ (A - 4 \ident) \vec{v} = \vec{0} \Rightarrow \left( \begin{bmatrix}
2 & 1\\
4 & 2
\end{bmatrix} 
- \begin{bmatrix}
4 & 0 \\
0 & 4
\end{bmatrix} \right)
\begin{bmatrix}
v_1 \\
v_2
\end{bmatrix}
= \begin{bmatrix}
0 \\
0
\end{bmatrix}
\]

\begin{align*}
&\Rightarrow \begin{bmatrix}[cc|c]
-2 & 1 & 0 \\
4 & -2 & 0
\end{bmatrix} \\
&\Rightarrow \begin{bmatrix}[cc|c]
-2 & 1 & 0 \\
0 & 0 & 0
\end{bmatrix}
\begin{matrix}
\\
R_2 \rightarrow R_2 + 2R_1
\end{matrix} \\
&\Rightarrow -2v_1 + v_2 = 0
\end{align*}
Setting $v_1 = 1$, shows that
\[ \vec{v_1} = \begin{bmatrix}
1 \\
2
\end{bmatrix} \]
is an eigenvector of $A$ with eigenvalue $\lambda_1 = 4$.

\underline{Now, the eigenvector associated with $\lambda = 0$:}

\begin{align*}
(A - 0 \ident) \vec{v} - \vec{0} &\Rightarrow \begin{bmatrix}
2 & 1 \\
4 & 2
\end{bmatrix}
\begin{bmatrix}
v_1 \\
v_2
\end{bmatrix}
= \begin{bmatrix}
0 \\
0
\end{bmatrix} \\
&\Rightarrow \begin{bmatrix}[cc|c]
2 & 1 & 0 \\
4 & 2 & 0
\end{bmatrix} \\
&\Rightarrow \begin{bmatrix}[cc|c]
2 & 1 & 0 \\
0 & 0 & 0
\end{bmatrix}
\begin{matrix}
\\
R_2 \rightarrow R_2 - 2R_1
\end{matrix} \\
\Rightarrow 2v_1 + v_2 = 0
\end{align*}
Setting $v_1 = 1$, shows that
\[ \vec{2} = \begin{bmatrix}
1 \\
-2
\end{bmatrix}
\]
is an eigenvector of $A$ with eigenvalue $\lambda_2 = 0$.

\section{Complete set of eigenvectors §5.7.2, 5.7.7}
\subsection{Definition}
An $n \times n$ matrix $A$ has a complete set of eigenvectors if there exists $n$ linearly independent eigenvectors $v_1, v_2, ..., v_n$. That is, 
\[ \alpha_1 \vec{v}_1 + \alpha_2 \vec{v}_2 + ... + \alpha_n \vec{v}_n = \vec{0} \Leftrightarrow \alpha_1 = \alpha_2 =...=\alpha_n = 0 \]

\subsection{Theorem}
Suppose $A$ is an $n \times n$ matrix.

\begin{enumerate}[ (a) ]
\item If $A$ has $n$ distinct eigenvalues, then $A$ has a complete set of eigenvectors.
\item If $A$ is a \textbf{symmetric} matrix, ($A^T = A$), then $A$ has a complete set of \textbf{orthonormal} eigenvectors $\vec{v}_1, \vec{v}_2, ..., \vec{v}_n$ with associated \textbf{real} eigenvalues.

Note that orthonormal means
\[ \vec{v}_i \cdot \vec{v}_j = \delta_{ij} \quad , 1 \leq i,\ j \leq n\]
Note that $[\delta{ij}] = \ident$.
\end{enumerate}

\subsection{Remark}
If $\vec{v}_1, \vec{v}_2, ..., \vec{v}_n$ is a complete set of eigenvectors for an $n \times n$ matrix $A$, then any $n$-column vector $\vec{w}$ can be expressed as
\[ \vec{w} = \sum_{j=1}^n c_j \vec{v}_j \]

Using this, we see
\[ A\vec{w} = \sum_{j=1}^n c_j A \vec{v}_j = \sum_{j=1}^n c_j \lambda \vec{v}_j \]


\end{document}