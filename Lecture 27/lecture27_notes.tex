\documentclass[11pt]{article}

\usepackage{amsmath}
\usepackage{amsfonts}
\usepackage{amssymb}

% Give ourself extra space for text
\usepackage[left = 2.2cm, right = 2.2cm, top = 1.8cm, bottom = 2.8cm]{geometry}

% Allows us to easily change the numbering system used in things like \begin{enumerate}. https://ctan.org/tex-archive/macros/latex/contrib/enumitem/
\usepackage[shortlabels]{enumitem}

% Turns table of contents, \refs, etc. into hyperlinks
\usepackage{hyperref}

% To include image, just use \includegraphics[scale=•]{relative path to image}
\usepackage{graphicx}
\graphicspath{ {./images/} }

% Lets us do \nth{1}, \nth{2}, which gives 1st 2nd, etc.
\usepackage[super]{nth}

% Common sets
\newcommand{\integers}{\mathbb{Z}}
\newcommand{\naturals}{\mathbb{N}}
\newcommand{\reals}{\mathbb{R}}

% Power set
\newcommand{\powerset}{\mathcal{P}}

% Identity matrix
\newcommand{\ident}{\mathbb{I}}

% Inverse hyperbolic functions
\DeclareMathOperator{\arcosh}{arcosh}
\DeclareMathOperator{\arsinh}{arsinh}
\DeclareMathOperator{\artanh}{artanh}

% I hat, J hat, K hat
\newcommand{\ihat}{\boldsymbol{\hat{\textbf{\i}}}}
\newcommand{\jhat}{\boldsymbol{\hat{\textbf{\j}}}}
\newcommand{\khat}{\boldsymbol{\hat{\textbf{k}}}}

% Better vectors (for single characters)
\renewcommand{\vec}[1]{\mathbf{#1}}

% Allows us to number equations in \begin{align} statements, etc.
\newcommand\numberthis{\addtocounter{equation}{1}\tag{\theequation}}

% Augmented matrices: this allows us to make augmented matrics using something like \begin{bmatrix}[cc|c]. Taken from Stefan Kottwitz at https://tex.stackexchange.com/questions/2233/whats-the-best-way-make-an-augmented-coefficient-matrix.
\makeatletter
\renewcommand*\env@matrix[1][*\c@MaxMatrixCols c]{%
  \hskip -\arraycolsep
  \let\@ifnextchar\new@ifnextchar
  \array{#1}}
\makeatother

% NOTE: This means \section does NOT number sections, but ensures that they appear in the table of contents, which does not occur if simply \section* is used. From egreg @ https://tex.stackexchange.com/a/30225.
\setcounter{secnumdepth}{0} % sections are level 1

\begin{document}
\title{ENG1005: Lecture 27}
\author{Lex Gallon}
\maketitle

\tableofcontents

\section*{Video link}
\url{https://echo360.org.au/lesson/G_35fe23e0-41ee-4e6f-b0f5-05f4155bb7b0_b944cecf-8ba5-40d3-a870-0243a0a9e78c_2020-05-21T15:58:00.000_2020-05-21T16:53:00.000/classroom#sortDirection=desc}

\subsection{Example}
The temperature at every point in the closed unit disc
\[ D = \{ (x, y) \in \reals^2 | x^2 + y^2 \leq 1 \]
is given by
\[ T(x, y) = (x + y)e^{-(x^2 + y^2)} \]
Find the maximum and minimum temperature and where these are achieved on the disc.

\subsection{Solution}
First note that this is a nice continuous function. 

\underline{Step 1:} Find all the critical points of $T(x, y)$ in the interior of $D$.

\begin{align*}
\nabla T(x, y) = \vec{0} &\Rightarrow \left\{ \begin{matrix}
\frac{\partial T}{\partial x} = (1 - 2x(x+y))e^{-(x^2+y^2)} = 0\\
\frac{\partial T}{\partial y} = (1 - 2y(x+y))e^{-(x^2+y^2)} = 0\\
\end{matrix} \right. \\
&\Rightarrow \left\{ \begin{matrix}
\frac{\partial T}{\partial x} = 1 - 2x(x+y) = 0\\
\frac{\partial T}{\partial y} = 1 - 2y(x+y) = 0\\
\end{matrix} \right. \\
&\Rightarrow \left\{ \begin{matrix}
\frac{\partial T}{\partial x} = 2x(x+y) = 1\\
\frac{\partial T}{\partial y} = 2y(x+y) = 1\\
\end{matrix} \right. \\
\end{align*}

I'm gonna solve this by noting that $x=y$ as you can see that  both equations are very symmetric. Therefore, from either equation we get
\begin{align*}
2x ( x + x) &= 1  \\
4x^2 &= 1 \\
\Rightarrow x &= \pm \frac{1}{2}
\end{align*}
Therefore the critical points are 
\[ \left(-\frac{1}{2}, -\frac{1}{2}\right),\ \left( \frac{1}{2}, \frac{1}{2} \right) \]
Note also that both are inside $D$.

\underline{Step 2:} Find all the critical points of $T(x, y)$ on the boundary of $D$. We can parametrise the boundary of $D$ by 
\[ \vec{r}(t) = (\cos(t), \sin(t)),\ 0 \leq t \leq 2\pi \]
Restricting $T(x, y)$ to the boundary gives
\begin{align*}
g(t) &= T(\cos(t), \sin(t) ),\ 0 \leq t \leq 2\pi \\
&= (\cos(t) + \sin(t))e^{-1},\ 0 \leq t \leq 2\pi \\
g'(t) &= \frac{1}{e}(-\sin(t) + \cos(t)) = 0 \\
&\Rightarrow \tan(t) = 1 \\
&\Rightarrow t = \frac{\pi}{4}, \frac{5\pi}{4}
\end{align*}

The critical points of $T(x,y)$ restricted to the boundary of $D$ are
\[ \left( \frac{1}{\sqrt{2}}, \frac{1}{\sqrt{2}} \right),\ \left( -\frac{1}{\sqrt{2}}, -\frac{1}{\sqrt{2}} \right) \]

\underline{Step 3:} Evaluate $T(x,y)$ at all the critical points.

\begin{center}
\begin{tabular}{c|c|c}
Critical points & $T(x, y)$ & Conclusion \\
\hline
$(\frac{1}{2}, \frac{1}{2})$ & $\frac{1}{\sqrt{e}}$ & Max temperature \\ 
$(-\frac{1}{2}, -\frac{1}{2})$ & $-\frac{1}{\sqrt{e}}$ & Min temperature \\
$(\frac{1}{\sqrt{2}}, \frac{1}{\sqrt{2}})$ & $\frac{\sqrt{2}}{e}$ \\
$(-\frac{1}{\sqrt{2}}, -\frac{1}{\sqrt{2}})$ & $-\frac{\sqrt{2}}{e}$
\end{tabular}
\end{center}

\section{Ordinary differential equations §10.1, 10.2, 10.3}
An ordinary differential equation (ODE) is an equation that involves a function $y(x)$ and one or more of its derivatives.

The order of an ODE is the highest numbe of derivatives that appear in the equation.

An nth order ODE is called linear if is of the form
\[ \sum_{k=0}^n a_k(x) \frac{d^k y(x)}{dx^k} = q(x) \]
If an ODE is not linear, then it called non-linear (shock!).

\subsection{Examples}
\begin{enumerate}[ (i) ]
\item $ m \dfrac{d^2 x}{dt^2} = -kx $ governs the motion of a particle of mass $m$ that is attached to a spring. It's a 2nd order linear ODE.

\item $\frac{d^2h}{dt^2} + k \left( \dfrac{dh}{dt} \right)^2 - g = 0$. This roughly describes the motion of a parachute.
\begin{align*}
m &= \text{ mass of box and parachute} \\
h(t) &= \text{ height above the ground at time} \\
\rho &= \text{ density of air, } C_d \equiv \text{ dry coefficient } \\
g &= \text{ acceleration due to gravity } \\
k &= \frac{\pi \rho C_d D^2}{8m}
\end{align*}
Note this a 2nd order non-linear ODE>

\end{enumerate}

\section{Solutions of ODES §10.4}
A solution of an ODE is any function $y(x)$ that satisfies the equation.

\subsection{Example}
Verify that
\[ y(x) = \sqrt{1 - x^2} \]
solves the 1st order non-linear ODE
\[ \frac{dy}{dx} + \frac{x}{y} = 0 \]

\subsection{Solution}
\begin{align*}
\frac{dy}{dx} + \frac{x}{y} &= \frac{-2x}{2\sqrt{1-x^2}} + \frac{x}{\sqrt{1-x^2}} \\
&= \frac{-x}{\sqrt{1-x^2}} + \frac{x}{\sqrt{1-x^2}} \\
&= 0
\end{align*}

\section{General and particular solutions §10.4.2}
We say that a function $y=y(x, \alpha_1, \alpha_2, ..., \alpha_n)$ that on parameters $\alpha_1, \alpha_2, ..., \alpha_n$ is a \textbf{general solution} of an ODE if every solution of the ODE given by $y(x, \alpha_1, \alpha_2, ..., \alpha_n)$ for some choice of the parameters $\alpha_1, \alpha_2, ..., \alpha_n$.\\
A \textbf{particular solution} is any solution of the ODE.

\subsection{Example}
The general solution of the ODE
\[ \frac{d^2x}{dt^2} + x = 0 \]
is
\[ x(t) = A\sin(t) + B\cos(t),\ A, B \in \reals \]

So then setting $A = B = 1$, we get the particular solution
\[ x(t) = \sin(t) + \cos(t) \]

\subsection{Remark}
One generally expects that the general solution of an $n$th order ODE will depened on $n$ parameters.

\section{Boundary value problems (BVP) §10.4.3}
\subsection{Example}
Solve the BVP:
\[ \frac{dy^2}{dx^2} + y = 0,\ 0 < x < \frac{\pi}{2} \]
\[ y(0) = -1,\ y \left(\frac{\pi}{2} \right) = 2 \]

\subsection{Solution}
The general solution to
\[ \frac{d^2y}{dx^2} + y = 0 \]
is
\[ x(t) = A\sin(x) + B\cos(x),\ A, B \in \reals \]
The boundary conditions (BCs) then imply
\[ y(0) = B = -1 \text{ and } y\left(\frac{\pi}{2}\right) = A=2\]
so our solution is
\[ y(x) = 2\sin(x) - \cos(x) \]

\section{Initial value problems (IVP) §10.4.3}
\subsection{Example}
Newton's law of cooling states that the temperature of a homogeneous object satisfies
\[ \frac{dT}{dt} = -K(T - T_a) \]
where $T_a$ is the ambient temperature, $T(t)$ is temperature of body at time $t$, $K > 0$ is some decay constant. The initial condition is
\[ T(t_0) = T_0 \]

Find the temperature of the body at time $t > t_0$ assuming that $K > 0$ and $T_0 > T_a$.

\end{document}