\documentclass[11pt]{article}

\usepackage{amsmath}
\usepackage{amsfonts}
\usepackage{amssymb}

% Give ourself extra space for text
\usepackage[left = 2.2cm, right = 2.2cm, top = 1.8cm, bottom = 2.8cm]{geometry}

% Allows us to easily change the numbering system used in things like \begin{enumerate}. https://ctan.org/tex-archive/macros/latex/contrib/enumitem/
\usepackage[shortlabels]{enumitem}

% Turns table of contents, \refs, etc. into hyperlinks
\usepackage{hyperref}

% To include image, just use \includegraphics[scale=•]{relative path to image}
\usepackage{graphicx}
\graphicspath{ {./images/} }

% Common sets
\newcommand{\integers}{\mathbb{Z}}
\newcommand{\naturals}{\mathbb{N}}
\newcommand{\reals}{\mathbb{R}}

% Power set
\newcommand{\powerset}{\mathcal{P}}

% Inverse hyperbolic functions
\DeclareMathOperator{\arcosh}{arcosh}
\DeclareMathOperator{\arsinh}{arsinh}
\DeclareMathOperator{\artanh}{artanh}

% I hat, J hat, K hat
\newcommand{\ihat}{\boldsymbol{\hat{\textbf{\i}}}}
\newcommand{\jhat}{\boldsymbol{\hat{\textbf{\j}}}}
\newcommand{\khat}{\boldsymbol{\hat{\textbf{k}}}}

% Better vectors (for single characters)
\renewcommand{\vec}[1]{\mathbf{#1}}

% Allows us to number equations in \begin{align} statements, etc.
\newcommand\numberthis{\addtocounter{equation}{1}\tag{\theequation}}

% Augmented matrices: this allows us to make augmented matrics using something like \begin{bmatrix}[cc|c]. Taken from Stefan Kottwitz at https://tex.stackexchange.com/questions/2233/whats-the-best-way-make-an-augmented-coefficient-matrix.
\makeatletter
\renewcommand*\env@matrix[1][*\c@MaxMatrixCols c]{%
  \hskip -\arraycolsep
  \let\@ifnextchar\new@ifnextchar
  \array{#1}}
\makeatother

% NOTE: This means \section does NOT number sections, but ensures that they appear in the table of contents, which does not occur if simply \section* is used. From egreg @ https://tex.stackexchange.com/a/30225.
\setcounter{secnumdepth}{0} % sections are level 1

\begin{document}
\title{ENG1005: Lecture 21}
\author{Lex Gallon}
\maketitle

\tableofcontents

\section*{Video link}
Click \href{https://echo360.org.au/lesson/G_35fe23e0-41ee-4e6f-b0f5-05f4155bb7b0_b944cecf-8ba5-40d3-a870-0243a0a9e78c_2020-05-07T15:58:00.000_2020-05-07T16:53:00.000/classroom#sortDirection=desc}{here} for a recording of the lecture.

\section{Parametric surfaces - continued}
\subsection{Examples}
\begin{enumerate}[ (a) ]
\item $f(x,y) = \sqrt{1 - x^2 - y^2},\ x^2 + y^2 \leq 1$.\\
The graph of $f$ is the (unit) hemisphere.\\
$S = \{ (x, y, \sqrt{1 - x^2 - y^2})\ |\ x^2 + y^2 \leq 1 \}$.

\item $\vec{r}(\theta, z) = (\cos \theta, \sin \theta, z),\quad 0 \leq \theta \leq 2\pi,\ -\infty < z < \infty$.\\
This is a cylinder centred on the $z$-axis. Is this a graph?
\end{enumerate}

\section{Level surface}
Given a function of 3 variable $f(x,y,)$, we can define a surface
\[ S = \{ (x, y, z) \in \reals^3\ |\ f(x, y, z) = c \} \]
This type of surface is known as a level surface (note however that there can be some degenerate cases).

\subsection{Examples}
\begin{enumerate}[ (a) ]
\item Let $f(x, y, z) = x + 2y + 3z - 1$.\\
Then $f(x, y, z) = 0$ defines a plane in $\reals^3$.

\item Let $f(x, y, z) = x^2 + y^2 + z^2$.\\
Then $f(x, y, z) = 1 \Leftrightarrow x^2 + y^2 + z^2 = 1$\\
defines a unit sphere in $\reals^3$.

\item If $g(x, y)$ is a function of 2 variables, then the graph of $g(x ,y)$ can be represented as a level surface by setting 
\[ f(x ,y ,z) = g(x, y) - z \]
because then
\[ f(x, y, z) = 0 \Leftrightarrow z = g(x, y) \]

\item Let $f(x, y) =  x^2 + y^2 - 1$. Then
\[ f(x, y) = 0 \Leftrightarrow x^2 + y^2 = 1 \]
defines the unit circle in $\reals^2$.
\end{enumerate}

\section{Limits and continuity}
\subsection{Definition}
Given a function $f(\vec{x})$ of $n$ variables ($\vec{x} = (x_1, x_2, ..., x_n)$) that is defined for all $\vec{x}$ satisfying $0 < |\vec{x} - \vec{p}| < R$, we say that the limit of $f(\vec{x})$ equals $\ell$ as $\vec{x}$ goes to $\vec{p}$, denoted by
\[ \lim_{\vec{x} \rightarrow \vec{p}} f(\vec{x}) - \ell \]
if for every $\epsilon > 0$, there exists a $\delta \in (0, R)$ such that $|f(\vec{x}) - \ell| < \epsilon$ whenever $0 < |\vec{x} - \vec{p}| < \delta$.

Moreover,
\begin{enumerate}[ (i) ]
\item If $f(\vec{x})$ is defined for $\vec{x} = \vec{p}$ and $\displaystyle{ \lim_{\vec{x} \rightarrow \vec{p}} f(\vec{x}) = f(\vec{p}) }$, then we say that $f(\vec{x})$ is continuous at $\vec{p}$.

\item If $D \subset \reals^n$ and $f(\vec{x})$ is continuous at each $\vec{p}\in D$, then we say that $f(\vec{x})$ is continuous on $D$.
\end{enumerate}

\section{Our notation for balls}
\[ B_R(\vec{p}) = \{ \vec{x} \in \reals^n\ |\ |\vec{x} - \vec{p}| < R \} \]
\[ B_R^X(\vec{p}) = \{ \vec{x} \in \reals^n\ |\ 0 < |\vec{x} - \vec{p}| < R \} \quad (\text{doesn't include center point}) \]
\[ \overline{B}_R(\vec{p}) = \{ \vec{x} \in \reals^n\ |\ |\vec{x} - \vec{p}| \leq R \} \]
Note, $B$ stands for `ball', $R$ stands for `radius'.

\section{Partial derivatives §9.6.2}
\subsection{Definition}
The first partial derivatives of a two variable function $f(x, y)$ with respect to the variables $x$ and $y$ at the point $(a, b)$ are defined by
\[ \frac{\partial f}{\partial x}(a, b) = \lim_{h \rightarrow 0} \frac{f(a + h, b) - f(a, b)}{h} \]
and similarly
\[ \frac{\partial f}{\partial y}(a, b) = \lim_{h \rightarrow 0} \frac{f(a, b + h) - f(a, b)}{h} \]

\subsection{Remarks}
\begin{enumerate}[ (i) ]
\item If we define 
\[ h(x) = f(x, y),\ y\text{-fixed}, \quad \quad g(y) = f(x, y),\ x\text{-fixed} \]
then
\[ \frac{\partial f}{\partial x}(x, y) = \frac{dh}{dx}(x) \text{ and } \frac{\partial f}{\partial y}(x, y) = \frac{dg}{dy}(y) \]

\item There exist many notations for partial derivatives
\[ \frac{\partial f}{\partial x}(a, b) = f_x(a, b) = \partial_x f(a, b) = D_1 f(a, b) = \partial_1 f(a, b) \]
\end{enumerate}

\subsection{Example}
\[ f(x, y) = \cos(x) \sin(y) \]
Compute
\[ \frac{\partial f}{\partial x} (x, y) \]

\subsection{Solution}
Treating $y$ as a constant, then
\[ \frac{\partial}{\partial x}(\cos(x) \sin(y)) = \sin(y) \frac{d}{dx}(\cos(x)) = -\sin(y) \sin(x) \]
Therefore
\[ \frac{\partial f}{\partial x}(x, y) = -\sin(y) \sin(x) \]

\subsection{Example}
Compute
\[ \frac{\partial}{\partial x} \left( y^{yx^2} \right) \]

\subsection{Solution}
\begin{align*}
\frac{\partial}{\partial x} \left( y^{yx^2} \right) &= \ln(y)a^{yx^2} \frac{\partial}{\partial x} \left( yx^2 \right) \\
&= \ln(y)a^{yx^2} y \frac{d}{dx} \left( x^2 \right) \\
&= 2 \ln(y)a^{yx^2} y
\end{align*}

\section{Successive differentiation §9.6.7}
Given a function $f(x, y)$, then are 4 partial derivatives of second order.

\begin{align*}
\frac{\partial^2 f}{\partial x^2} &:= \frac{\partial}{\partial x}\left( \frac{\partial f}{\partial x} \right), & \frac{\partial^2 f}{\partial x \partial y} &:= \frac{\partial}{\partial x}\left( \frac{\partial f}{\partial y} \right) \\
\frac{\partial^2 f}{\partial y^2} &:= \frac{\partial}{\partial y}\left( \frac{\partial f}{\partial y} \right), & \frac{\partial^2 f}{\partial y \partial x} &:= \frac{\partial}{\partial y}\left( \frac{\partial f}{\partial x} \right)
\end{align*}

\end{document}