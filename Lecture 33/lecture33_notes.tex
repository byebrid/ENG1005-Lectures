\documentclass[11pt]{article}

\usepackage{amsmath}
\usepackage{amsfonts}
\usepackage{amssymb}

% Give ourself extra space for text
\usepackage[left = 2.2cm, right = 2.2cm, top = 1.8cm, bottom = 2.8cm]{geometry}

% Allows us to easily change the numbering system used in things like \begin{enumerate}. https://ctan.org/tex-archive/macros/latex/contrib/enumitem/
\usepackage[shortlabels]{enumitem}

% Turns table of contents, \refs, etc. into hyperlinks
\usepackage{hyperref}

% To include image, just use \includegraphics[scale=•]{relative path to image}
\usepackage{graphicx}
\graphicspath{ {./images/} }

% Common sets
\newcommand{\integers}{\mathbb{Z}}
\newcommand{\naturals}{\mathbb{N}}
\newcommand{\reals}{\mathbb{R}}
\newcommand{\complex}{\mathbb{C}}

% Power set
\newcommand{\powerset}{\mathcal{P}}

% Identity matrix
\newcommand{\ident}{\mathbb{I}}

% Inverse hyperbolic functions
\DeclareMathOperator{\arcosh}{arcosh}
\DeclareMathOperator{\arsinh}{arsinh}
\DeclareMathOperator{\artanh}{artanh}

% I hat, J hat, K hat
\newcommand{\ihat}{\boldsymbol{\hat{\textbf{\i}}}}
\newcommand{\jhat}{\boldsymbol{\hat{\textbf{\j}}}}
\newcommand{\khat}{\boldsymbol{\hat{\textbf{k}}}}

% Purely for Laplace symbol
\usepackage[scr]{rsfso}
\newcommand{\Laplace}{\mathscr{L}}

% Better vectors (for single characters)
\renewcommand{\vec}[1]{\mathbf{#1}}

% Allows us to number equations in \begin{align} statements, etc.
\newcommand\numberthis{\addtocounter{equation}{1}\tag{\theequation}}

% Augmented matrices: this allows us to make augmented matrics using something like \begin{bmatrix}[cc|c]. Taken from Stefan Kottwitz at https://tex.stackexchange.com/questions/2233/whats-the-best-way-make-an-augmented-coefficient-matrix.
\makeatletter
\renewcommand*\env@matrix[1][*\c@MaxMatrixCols c]{%
  \hskip -\arraycolsep
  \let\@ifnextchar\new@ifnextchar
  \array{#1}}
\makeatother

% NOTE: This means \section does NOT number sections, but ensures that they appear in the table of contents, which does not occur if simply \section* is used. From egreg @ https://tex.stackexchange.com/a/30225.
\setcounter{secnumdepth}{0} % sections are level 1

\begin{document}
\title{ENG1005: Lecture 33}
\author{Lex Gallon}
\maketitle

\tableofcontents

\section*{Video link}
\url{https://echo360.org.au/lesson/G_35fe23e0-41ee-4e6f-b0f5-05f4155bb7b0_b944cecf-8ba5-40d3-a870-0243a0a9e78c_2020-06-04T15:58:00.000_2020-06-04T16:53:00.000/classroom#sortDirection=desc}

\section{Laplace transforms of integrals §11.3.2}
Let
\[ g(t) = \int_0^t f(\tau)\, d\tau. \]
Then, by the Fundamental Theorem of Calculus, we have
\[\frac{dg}{dt}(t) = f(t).  \]
So
\[ \Laplace[f(t)](s) = \Laplace \left[ \frac{dg}{dt}(t) \right](s) = s\Laplace[g(t)](s) - g(0) = s\Laplace \left[ \int_0^t f(\tau)\, d\tau \right](s). \]
It follows that
\[ \Laplace \left[ \int_0^t f(\tau)\, d\tau \right](s) = \frac{1}{s} \Laplace[f(t)](s). \]

\section{The inverse Laplace Transform §11.2.7}
If $f(t)$ and $g(t)$ both satisfy 
\[ |f(t)| \leq Me^{\sigma t} \text{ and } |g(t)| \leq Me^{\sigma t},\ t \geq 0, \]
for some $M \geq 0$ and $\sigma \in \reals$, and
\[ \Laplace[f(t)](s) = \Laplace[g(t)](s),\ \text{Re}(s) > \sigma. \]
Then it can be shown that
\[ f(t) = g(t),\ t \geq 0. \]
This allows us to uniquely define the inverse Laplace transform by
\[ \Laplace^{-1}[\Laplace[f(t)]] = H(t)f(t), \]
where $H(t)$ is the Heaviside step function
\[ H(t) = \begin{cases}
1 &\text{if } t \geq 0, \\
0 &\text{if } t < 0
\end{cases} \]

\subsection{Bronwich integral (not examinable)}
The inverse Laplace transform can be computed using
\[ f(t) = \Laplace^{-1}[F(s)](t) := \frac{1}{2\pi i} \lim_{T \rightarrow \infty } \int_{\sigma -iT}^{\sigma + iT} F(s)\, ds \]
where all singularities of $F(s)$ lie in the region $\text{Re}(s) > \sigma$.

\subsection{Example}
Compute
\[ \Laplace^{-1}\left[ \frac{s+1}{s(s^2 + 2)} \right](t) \]

\subsection*{Solution}
Use partial fractions, we know that
\begin{align*}
\frac{s+1}{s(s^2 + 1)} &= \frac{1}{2}\frac{1}{s} + \frac{1 - \frac{1}{2}s}{s^2 + 2} \\
&= \frac{1}{2}\frac{1}{s} + \frac{1}{\sqrt{2}}\frac{\sqrt{2}}{s^2 + (\sqrt{2})^2} - \frac{1}{2}\frac{s}{s^2 + (\sqrt{2})^2} \\
&= \Laplace \left[ \frac{1}{2} + \frac{1}{\sqrt{2}}\sin(\sqrt{2}t) - \frac{1}{2}\cos(\sqrt{2}t) \right](s)
\end{align*}
So then
\[ \Laplace^{-1}\left[ \frac{s+1}{s(s^2 + 2)} \right](t) = \frac{1}{2} + \frac{1}{\sqrt{2}}\sin(\sqrt{2}t) - \frac{1}{2}\cos(\sqrt{2}t) \]

\section{Inverse of the s-shift property §11.2.9}
\subsection{Theorem}
Suppose $f(t)$, $t \geq 0$, has a Laplace Transform
\[ F(s) = \Laplace[f(t)](s),\ \text{Re}(s) > \sigma, \]
and $a \in \reals$. Then
\[ \Laplace^{-1}[e^{-as}F(s)](t) = H(t-a)f(t-a). \]

\subsection{Proof}
By definition
\begin{align*}
\Laplace[H(t-a)f(t-a)](s) &= \int_0^\infty H(t-a)f(t-a)\, dt \\
&= \int_{-a}^\infty e^{-s(\tau + a)}H(\tau)f(\tau)\, d\tau,\quad , \tau = t-a \\
&= \int_0^\infty e^{-s(\tau + a)}f(\tau)\, d\tau,\quad , \tau = t-a \\
&= \int_0^\infty e^{-s\tau}e^{-sa}f(\tau)\, d\tau,\quad , \tau = t-a \\
&= e^{-sa} \int_0^\infty e^{-s\tau}f(\tau)\, d\tau,\quad , \tau = t-a \\
&= e^{-sa} \Laplace[f(t)](s) \\
\Rightarrow \Laplace[H(t-a)f(t-a)](s) &= e^{-sa} \Laplace[f(t)](s) = e^{-sa}F(s) \\
\Rightarrow H(t-a)f(t-a) &= \Laplace^{-1}[e^{-sa}F(s)](t)
\end{align*}

\section{Convolutions}
The convolution of two functions $f(t)$ and $g(t)$, $t \geq 0$, denoted by $f*g(t)$, is the function defined by
\[ (f*g)(t) = \int_0^t f(\tau) g(t - \tau)\, d\tau . \]

\subsection{Theorem}
Suppose $f(t)$ and $g(t)$ are two functions satisfying
\[ |f(t)| \leq Me^{\sigma t} \text{ and } |g(t)| \leq Me^{\sigma t},\ t \geq 0 \]
for some $M \geq 0$, and $\sigma \in \reals$. Then
\[ \Laplace[(f*g)(t)](s) = F(s)G(s),\ \text{Re}(s) > \sigma, \]
where
\[ F(s) = \Laplace[f(t)](s) \text{ and } G(s) = \Laplace[g(t)](s) . \]

\subsection{Example}
Use the Laplace transform to show that
\[ \int_0^t t\sin(t - \tau) \, d\tau = t - \sin(t). \]
Note you could probably do integration by parts but whatever.

\subsection*{Solution}
\begin{align*}
\Laplace \left[ \int_0^t t\sin(t - \tau) \, d\tau \right](s) &= \Laplace[t * \sin(t)](s) \\
&= \Laplace[t](s) \Laplace[\sin(t)](s) \\
&= \frac{1}{s^2}\frac{1}{s^2 + 1} \\
&= \frac{1}{s^2} - \frac{1}{s^2+1} \\
\Rightarrow \int_0^t t\sin(t - \tau) \, d\tau &= \Laplace^{-1} \left[ \frac{1}{s^2} - \frac{1}{s^2+1} \right](t) \\
&= \Laplace^{-1}\left[ \frac{1}{s^2} \right](t) - \Laplace^{-1}\left[ \frac{1}{s^2 + 1} \right](t) \\
&= t - \sin(t)
\end{align*}

\section{Impulse}
Mathematically, a short, sharp forcing at $t=0	$ is represented by the Dirac Delta Function (although it's not really a function), denoted by $\delta(t)$, which is defined by
\[ \delta(t) = 0,\ t=0, \]
and
\[ \int_a^b f(t) \delta(t)\, dt = \begin{cases}
f(0) &\text{if } 0 \in (a, b) \\
0 &\text{otherwise} 
\end{cases} \]
for all continuous functions $f(t)$ that are defined on $[a, b]$.

\subsection{Example}
Compute the Laplace transform of
\[ \delta(t - a),\ a > 0. \]

\subsection*{Solution}
\[ \Laplace[\delta(t - a)](s) = \int_0^\infty e^{-st} \delta(t - a)]\, dt = e^{-sa}. \]
\end{document}