\documentclass[11pt]{article}

\usepackage{amsmath}
\usepackage{amsfonts}
\usepackage{amssymb}

% Give ourself extra space for text
\usepackage[left = 2.2cm, right = 2.2cm, top = 1.8cm, bottom = 2.8cm]{geometry}

% Allows us to easily change the numbering system used in things like \begin{enumerate}. https://ctan.org/tex-archive/macros/latex/contrib/enumitem/
\usepackage[shortlabels]{enumitem}

% Turns table of contents, \refs, etc. into hyperlinks
\usepackage{hyperref}

% To include image, just use \includegraphics[scale=•]{relative path to image}
\usepackage{graphicx}
\graphicspath{ {./images/} }

% Common sets
\newcommand{\integers}{\mathbb{Z}}
\newcommand{\naturals}{\mathbb{N}}
\newcommand{\reals}{\mathbb{R}}

% Power set
\newcommand{\powerset}{\mathcal{P}}

% Identity matrix
\newcommand{\ident}{\mathbb{I}}

% Inverse hyperbolic functions
\DeclareMathOperator{\arcosh}{arcosh}
\DeclareMathOperator{\arsinh}{arsinh}
\DeclareMathOperator{\artanh}{artanh}

% I hat, J hat, K hat
\newcommand{\ihat}{\boldsymbol{\hat{\textbf{\i}}}}
\newcommand{\jhat}{\boldsymbol{\hat{\textbf{\j}}}}
\newcommand{\khat}{\boldsymbol{\hat{\textbf{k}}}}

% Better vectors (for single characters)
\renewcommand{\vec}[1]{\mathbf{#1}}

% Allows us to number equations in \begin{align} statements, etc.
\newcommand\numberthis{\addtocounter{equation}{1}\tag{\theequation}}

% Augmented matrices: this allows us to make augmented matrics using something like \begin{bmatrix}[cc|c]. Taken from Stefan Kottwitz at https://tex.stackexchange.com/questions/2233/whats-the-best-way-make-an-augmented-coefficient-matrix.
\makeatletter
\renewcommand*\env@matrix[1][*\c@MaxMatrixCols c]{%
  \hskip -\arraycolsep
  \let\@ifnextchar\new@ifnextchar
  \array{#1}}
\makeatother

% NOTE: This means \section does NOT number sections, but ensures that they appear in the table of contents, which does not occur if simply \section* is used. From egreg @ https://tex.stackexchange.com/a/30225.
\setcounter{secnumdepth}{0} % sections are level 1

\begin{document}
\title{ENG1005: Lecture 22}
\author{Lex Gallon}
\maketitle

\tableofcontents

\section*{Video link}
Click \href{https://echo360.org.au/lesson/G_8402119b-734b-4e1e-a3b4-7e907e86ddba_b944cecf-8ba5-40d3-a870-0243a0a9e78c_2020-05-12T15:58:00.000_2020-05-12T16:53:00.000/classroom#sortDirection=desc}{here} for a recording of the lecture.

\section{Partial derivatives - continued}
\subsection{Example}
Compute all 2nd order partial derivatives for
\[ f(x, y) = e^{x^2y} \]

\subsection{Solution}
\begin{align*}
\frac{\partial^2 f}{\partial y \partial x} &=\frac{\partial}{\partial y} \left( \frac{\partial f}{\partial x} \right) \\
&= \frac{\partial}{\partial y} \left( 2xy e^{x^2y} \right) \\
&= 2x \frac{\partial}{\partial y} \left( y e^{x^2y} \right) \\
&= 2x \left( e^{x^2y} + y x^2 e^{x^2y} \right) \\
&= 2x e^{x^2y} + 2x^3y e^{x^2y}
\end{align*}

\begin{align*}
\frac{\partial^2 f}{\partial x^2} &= \frac{\partial}{\partial x} \left( \frac{\partial f}{\partial x} \right) \\
&= \frac{\partial}{\partial x} \left( 2xy e^{x^2y} \right) \\
&= 2y \frac{\partial}{\partial x} \left( x e^{x^2y} \right) \\
&= 2y \left( e^{x^2y} + x 2xy e^{x^2y} \right) \\
&= 2y e^{x^2y} + 4x^2y^2 e^{x^2y}
\end{align*}

\begin{align*}
\frac{\partial^2 f}{\partial x \partial y} &= \frac{\partial }{\partial x} \left( \frac{\partial f}{\partial y} \right) \\
&=  \frac{\partial }{\partial x} \left( x^2 e^{x^2 y} \right) \\
&= x^2 2xy e^{x^2y} + 2x e^{x^2y} \\
&= 2x^3 y e^{x^2y} + 2x e^{x^2y}
\end{align*}

\begin{align*}
\frac{\partial^2 f}{\partial y^2} &= \frac{\partial}{\partial y} \left( \frac{\partial f}{\partial y} \right) \\
&= \frac{\partial}{\partial y} \left( x^2 e^{x^2y} \right) \\
&= x^2 \frac{\partial}{\partial y} \left( e^{x^2y} \right) \\
&= x^2 \left( x^2 e^{x^2y} \right) \\
&= x^4 e^{x^2y}
\end{align*}

\subsection{Observation}
\[ \frac{\partial^2}{\partial x \partial y} \left( e^{x^2y} \right) = \frac{\partial^2}{\partial y \partial x} \left( e^{x^2y} \right) \]

\subsection{Theorem: Equality of mixed partials}
Suppose that there exists some $R>0$ such that $f(x,y), \dfrac{\partial f}{\partial x}, \dfrac{\partial f}{\partial y},  \dfrac{\partial^2 f}{\partial x \partial y}, \dfrac{\partial^2 f}{\partial y \partial x}$ are continuous on the ball $B_R((a, b))$. Then
\[ \frac{\partial^2 f}{\partial x \partial y}(a, b) = \frac{\partial^2 f}{\partial y \partial x}(a, b) \]

\subsection{Challenge}
Find a function $f(x, y)$ such that $\dfrac{\partial^2 f}{\partial x \partial y}$ and $\dfrac{\partial^2 f}{\partial y \partial x}$ both exist on some disc $B_R((a, b))$ in $\reals^2$, but 
\[ \frac{\partial^2 f}{\partial x \partial y}(a, b) \not = \frac{\partial^2 f}{\partial y \partial x}(a, b) \]

\section{The Chain Rule §9.6.5}
\subsection{Single variable chain rule}
Start with some differentiable functions
\[ f(x) \text{ and } x(t) \]
and define a new function
\[ y(t) = f(x(t)) \]
Then
\[ \frac{dy}{dt}(t) = \frac{df}{dx}\left( x(t) \right) \frac{dx}{dt}(t) \]
\[ \text{Alternatively, } \dfrac{df}{dt} = \dfrac{df}{dx} \dfrac{dx}{dt} \]

\subsection{Two variable chain rule}
\subsubsection{Theorem}
Suppose $f(x, y)$, $x(s, t)$ and $y(s, t)$ are all differentiable functions, and define
\[ z(s, t) = f(x(s,t), y(s,t)) \]
Then $z(s, t)$ is differentiable and
\[ \frac{\partial z}{\partial s}(s, t) = \frac{\partial f}{\partial x} \left( x(s, t), y(s, t) \right) \frac{\partial x}{\partial s}(s, t) + \frac{\partial f}{\partial y} \left( x(s, t), y(s, t) \right) \frac{\partial y}{\partial s}(s, t) \]
\[ \frac{\partial z}{\partial s}(s, t) = \frac{\partial f}{\partial x} \left( x(s, t), y(s, t) \right) \frac{\partial x}{\partial t}(s, t) + \frac{\partial f}{\partial y} \left( x(s, t), y(s, t) \right) \frac{\partial y}{\partial t}(s, t) \]

\subsubsection{Informal versions}
\[ \frac{\partial z}{\partial s} = \frac{\partial f}{\partial x} \frac{\partial x}{\partial s} + \frac{\partial f}{\partial y} \frac{\partial y}{\partial s} \]
\[ \frac{\partial z}{\partial t} = \frac{\partial f}{\partial x} \frac{\partial x}{\partial t} + \frac{\partial f}{\partial y} \frac{\partial y}{\partial t} \]
\[ \frac{\partial f}{\partial s} = \frac{\partial f}{\partial x} \frac{\partial x}{\partial s} + \frac{\partial f}{\partial y} \frac{\partial y}{\partial s} \]
\[ \frac{\partial f}{\partial t} = \frac{\partial f}{\partial x} \frac{\partial x}{\partial t} + \frac{\partial f}{\partial y} \frac{\partial y}{\partial t} \]

\subsection{Full Chain Rule}
\subsection{Theorem}
Suppose $f(x_1, x_2, ..., x_n)$ is differentiable and
\[ x_i(t_1, t_2, ..., t_m),\ i = 1, 2, ..., n \]
are differentiable. Then
\[ z(t_1, t_2, ..., t_m) = f(x_1(t_1, t_2, ..., t_m), x_2((t_1, t_2, ..., t_m), ..., x_n(t_1, t_2, ..., t_m)) \]
is differentiable and
\[ \frac{\partial z}{\partial t_j} (t_1, ..., t_m) = \sum_{i=1}^n \frac{\partial f}{\partial x_i}( x_1(t_1, ..., t_m), ..., x_n(t_1 ..., t_m) ) \frac{\partial x_i}{\partial t_j} (t_1, ..., t_m) \]

\subsubsection{Informal versions}
\[ \frac{\partial z}{\partial t_j} = \sum_{i=1}^n \frac{\partial f}{\partial x_i} \frac{\partial x_i}{\partial t_j} \text{ for } j = 1, 2, ..., m  \]

\subsection{Example}
Let $f(x, y) = \ln(x^2 + y^2),\ x(t) = \cos(t),\ y(s, t) = st$ and define
\[ z(s, t) = f(x(t), y(s, t)) = \ln \left( \cos^2(t) + s^2 t^2 \right) \]

\subsection{Solution}
\begin{align*}
\frac{\partial z}{\partial s} &= \frac{\partial f}{\partial x} \frac{\partial x}{\partial s} + \frac{\partial f}{\partial y} \frac{\partial y}{\partial s} \\
&= 0 + \frac{2y}{x^2 + y^2} t \\
&= \frac{2st^2}{\cos^2(t) + s^2t^2}
\end{align*}

\begin{align*}
\frac{\partial z}{\partial t} &= \frac{\partial f}{\partial x} \frac{\partial x}{\partial t} + \frac{\partial f}{\partial y} \frac{\partial y}{\partial t} \\
&= \frac{2x}{x^2 + y^2} (- \sin(t)) + \frac{2y}{x^2 + y^2} s \\
&= \frac{-2\cos(t) \sin(t)}{\cos^2(t) + s^2t^2} + \frac{2s^2t}{\cos^2(t) + s^2t^2} = \frac{2s^2t -2\cos(t) \sin(t)}{\cos^2(t) + s^2t^2}
\end{align*}

\end{document}