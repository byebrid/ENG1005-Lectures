\documentclass[11pt]{article}

\usepackage{amsmath}
\usepackage{amsfonts}
\usepackage{amssymb}

% Allows us to easily change the numbering system used in things like \begin{enumerate}. For usage, see https://ctan.org/tex-archive/macros/latex/contrib/enumitem/
\usepackage[shortlabels]{enumitem}

% Turns table of contents, \refs, etc. into hyperlinks
\usepackage{hyperref}

% Common sets
\newcommand{\integers}{\mathbb{Z}}
\newcommand{\naturals}{\mathbb{N}}
\newcommand{\reals}{\mathbb{R}}

% Inverse hyperbolic functions
\DeclareMathOperator{\arcosh}{arcosh}
\DeclareMathOperator{\arsinh}{arsinh}
\DeclareMathOperator{\artanh}{artanh}

% Allows us to number equations in \begin{align} statements, etc.
\newcommand\numberthis{\addtocounter{equation}{1}\tag{\theequation}}

% NOTE: This means \section does NOT number sections, but ensures that they appear in the table of contents, which does not occur if simply \section* is used. From egreg @ https://tex.stackexchange.com/a/30225.
\setcounter{secnumdepth}{0} % sections are level 1

\begin{document}
\title{ENG1005: Lecture 5}
\author{Lex Gallon}
\maketitle

\tableofcontents

\section{Comparison principle: divergence}
\subsection*{Example}
Determine if
\[ \int_2^\infty \frac{2+\sin(x)}{x} \, dx \]
is convergent or divergent.

\subsection*{Solution}
\[ \int_2^\infty \frac{2+\sin(x)}{x} \, dx = \lim_{\varepsilon\rightarrow\infty} \int_2^\varepsilon\frac{2+\sin(x)}{x} \, dx \]

But observe that
\[ 1 \leq 2 + \sin x,\ x \in \reals \]
which gives
\[ \frac{1}{x} \leq \frac{2 + \sin x}{x},\ x >0 \]

So then
\[ \int_2^\varepsilon\frac{1}{x} \, dx \leq \int_2^\varepsilon\frac{2+\sin(x)}{x} \, dx \]

while
\[
\lim_{\varepsilon\rightarrow\infty} \int_2^\varepsilon\frac{1}{x} \, dx = \lim_{\varepsilon\rightarrow\infty} \left. \ln(x)\right|_2^\varepsilon = \lim_{\varepsilon\rightarrow\infty} \left( ln(\varepsilon) - ln(2) \right) = \infty
\]

So we conclude by the Comparison Theorem/Test that 
\[ \lim_{\varepsilon\rightarrow\infty} \int_2^\varepsilon \frac{2+\sin x}{x} \,dx = \infty \]

\[ \text{Then the improper integral } \int_2^\infty \frac{2+\sin(x)}{x} \, dx \text{ DNE.} \]

\subsection*{Theorem (Comparison Test)}
\begin{enumerate}[(i)]
\item Suppose f(x), g(x) are continuous on $[a, \infty)$, $f(x) \leq g(x)$ for all $x \in [a, \infty)$, and
\[ \lim_{\varepsilon\rightarrow\infty} \int_a^\varepsilon f(x) \,dx = \infty \]
\[ \text{then } \int_a^\infty g(x) \,dx \text{ is divergent.} \]

\item Suppose f(x), g(x) are continuous on $(a, b]$, $f(x) \leq g(x)$ for all $x \in (a, b]$, and
\[ \lim_{\varepsilon\searrow a} \int_\varepsilon^b f(x) \,dx = \infty \]
\[ \text{then } \int_a^b g(x) \,dx \text{ is divergent.} \]
\end{enumerate}

\subsection*{Proof}
\begin{enumerate}[(i)]
\item If $f(x) \leq g(x)$ for all x $[a, \infty)$, then
\[ \int_a^\varepsilon f(x) \,dx \leq \int_a^\varepsilon g(x) \,dx = \infty for all \varepsilon > a \]

So if
\[ \lim_{\varepsilon \rightarrow \infty} \int_a^\varepsilon f(x) \,dx \leq \int_a^\varepsilon g(x) \,dx = \infty \]

then
\[ \lim_{\varepsilon \rightarrow \infty} \int_a^\varepsilon g(x) \,dx \leq \int_a^\varepsilon g(x) \,dx = \infty \]

\[ \text{This proves that } \int_a^b g(x) \,dx \text{ is divergent.} \]

\item Proof is similar to (i).

\end{enumerate}

\subsection*{Example}
For what values of $p \in \reals$ is $\int_1^\infty (2 + \sin x) x^p \,dx$ convergent?

\subsection*{Solution}
Since
\[ 1 \leq 2 + \sin x \leq 3, x \in \reals \]
we get
\[ x^p \leq x^p (2 + \sin x) \leq 3x^p, x \geq 0 \]

So then
\begin{equation} \label{inequality}
\int_1^\varepsilon x^p \,dx \leq \int_1^\varepsilon (2+ \sin x) x^p \, dx \leq 3 \int_1^\varepsilon x^p \,dx
\end{equation}

Since
\begin{align*}
\int_1^\varepsilon x^p \,dx &= 
\begin{cases}
	\left. \frac{x^{p+1}}{p+1} \right|_1^\varepsilon & p \neq -1\\ 
	\ln(x)|_1^\varepsilon & p = -1
\end{cases} \\
&=
\begin{cases}
	\left. \frac{\varepsilon^{p+1}}{p+1} - \frac{1}{p+1} \right|_1^\varepsilon & p \neq -1\\ 
	\ln(\varepsilon)|_1^\varepsilon & p = -1
\end{cases}
\end{align*}

We see
\begin{equation} \label{evaluation}
\lim_{\varepsilon \rightarrow \infty} \int_1^\varepsilon x^p \, dx = \begin{cases}
	\infty & p =-1 \\
	\infty & p + 1 > 0 <> p > -1 \\
	- \frac{1}{p+1} & p + 1 < 0 <> p < -1
\end{cases}
\end{equation}

We can conclude from (\ref{inequality}) and from (\ref{evaluation}) and the Comparison Test that 
\[ \int_1^\varepsilon (2+ \sin x) x^p \, dx \]
is convergent when $p < - 1$ and divergent when $p \geq -1$.

\subsection*{Example}
Determine if the improper integral
\[ \int_0^1 \frac{1}{1-x^4} \, dx \]
is convergent or divergent.

\subsection*{Solution}
Note that
\begin{align*}
1 - x^4 = (1-x^2)(1+x^2) = (1-x)(1+x)(1+x^2)
\end{align*}

So then
\begin{align*}
\frac{1}{(1-x)(1+x)(1+x^2)} &= \frac{1}{1-x^4} ,\ 0 \leq x < 1 \\
\frac{1}{1-x} \cdot \frac{1}{(1+x)(1+x^2)} &= \frac{1}{1-x^4} ,\ 0 \leq x < 1 \\
\end{align*}

This shows
\begin{align*}
\frac{1}{4} \cdot \frac{1}{1-x} \leq \frac{1}{1-x^4} ,\ 0 \leq x < 1
\end{align*}

Integrate to get
\begin{align*}
\int_0^\varepsilon \frac{1}{4} \cdot \frac{1}{1-x} \,dx &\leq \int_0^\varepsilon \frac{1}{1-x^4} \,dx \\
\left. -\frac{1}{4} \ln(1-x) \right|_0^\varepsilon \leq \int_0^\varepsilon \frac{1}{1-x^4} \,dx \\
-\frac{1}{4} \ln(1-\varepsilon) \leq \int_0^\varepsilon \frac{1}{1-x^4} \,dx \\ \numberthis \label{evaluation2}
\end{align*}

Now
\begin{equation} \label{inequality2}
\lim_{\varepsilon\nearrow 1} -\frac{1}{4} \ln(1-\varepsilon) = \infty
\end{equation}

and we conclude via the Comparison Test and equations (\ref{inequality2}) and (\ref{evaluation2}) that
\[ \int_0^1 \frac{1}{1-x^4} \, dx \text{ is divergent.} \]

\end{document}