\documentclass[11pt]{article}

\usepackage{amsmath}
\usepackage{amsfonts}
\usepackage{amssymb}

% Give ourself extra space for text
\usepackage[left = 2.2cm, right = 2.2cm, top = 1.8cm, bottom = 2.8cm]{geometry}

% Allows us to easily change the numbering system used in things like \begin{enumerate}. https://ctan.org/tex-archive/macros/latex/contrib/enumitem/
\usepackage[shortlabels]{enumitem}

% Turns table of contents, \refs, etc. into hyperlinks
\usepackage{hyperref}

% Common sets
\newcommand{\integers}{\mathbb{Z}}
\newcommand{\naturals}{\mathbb{N}}
\newcommand{\reals}{\mathbb{R}}

% Inverse hyperbolic functions
\DeclareMathOperator{\arcosh}{arcosh}
\DeclareMathOperator{\arsinh}{arsinh}
\DeclareMathOperator{\artanh}{artanh}

% Allows us to number equations in \begin{align} statements, etc.
\newcommand\numberthis{\addtocounter{equation}{1}\tag{\theequation}}

% NOTE: This means \section does NOT number sections, but ensures that they appear in the table of contents, which does not occur if simply \section* is used. From egreg @ https://tex.stackexchange.com/a/30225.
\setcounter{secnumdepth}{0} % sections are level 1

\begin{document}
\title{ENG1005: Lecture 9 - Taylor Series}
\author{Lex Gallon}
\maketitle

\tableofcontents

\section*{Video link}
Click \href{https://echo360.org.au/lesson/G_35fe23e0-41ee-4e6f-b0f5-05f4155bb7b0_b944cecf-8ba5-40d3-a870-0243a0a9e78c_2020-04-02T15:58:00.000_2020-04-02T16:53:00.000/classroom#sortDirection=desc}{here} for a recording of the lecture.

\section{Taylor's Theorem §9.4.1}
\subsection{Theorem (Taylor's Theorem)}
If the derivatives $f^{(k)}(x), k=1,2,...,n, n+1$ exist and are continuous on $[x_0 - r, x_0 + r], r>0$, then
\[ f(x) = \sum_{k=0}^n \frac{f^{(k)}(x_0)}{k!} (x-x_0)^k + (x-x_0)^{n+1} R_n(x), |x-x_0| \leq r \]
where $R_n(x)$ is continuous on $[x_0 - r, x_0 + r]$. We further note that for fixed $x \in [x_0 - r, x_0 + r], R_n(x)$ can be expressed as
\[ R_n(x) = \frac{1}{(n+1)!} f^{n+1} (x_0 + \theta(x-x_0)), 0 <\theta < 1 \]

Note that this $R_n(x)$ is a remainder of sorts.

So, informally we have that
\[ f(x) \approx p_n(x) := \sum_{k=0}^n \frac{f^{(k)}(x_0)}{k!} (x-x_0)  \]

\subsection{Example}
Approximate $\sin(x)$ by a 4th order Taylor polynomial near $x = \frac{\pi}{2}$.

\subsection{Solution}
Let
\[ f(x) = \sin(x),\ x_0=\frac{\pi}{2} \]

Then

\begin{tabular}{| c | c | c |}
\hline
$k$ & $f^k(x)$ & $f^k(\frac{\pi}{2}$ \\
\hline
0 & $\sin(x)$ & 1 \\
1 & $\cos(x)$ & 0 \\
2 & $-\sin(x)$ & -1 \\
3 & $-\cos(x)$ & 0 \\
4 & $\sin(x)$ & 1 \\
\hline
\end{tabular}

So the 4th order Taylor polynomial expansion of $f(x)$ about $x=\frac{\pi}{2}$ is
\begin{align*}
p_4(x) &= \sin(\frac{\pi}{2}) + \sin^{(1)}(\frac{\pi}{2}) (x-\frac{\pi}{2}) +  \frac{\sin^{(2)}(\frac{\pi}{2})}{2!} (x-\frac{\pi}{2})^2 +  \frac{\sin^{(3)}(\frac{\pi}{2})}{3!} (x-\frac{\pi}{2})^3 +  \frac{\sin^{(4)}(\frac{\pi}{2})}{4!} (x-\frac{\pi}{2})^4 \\
p_4(x) &= 1 + 0 + \frac{-1}{2} (x-\frac{\pi}{2})^2 + 0 + \frac{1}{4!} (x-\frac{\pi}{2})^4 \\
p_4(x) &= 1 + \frac{-1}{2} (x-\frac{\pi}{2})^2 + \frac{1}{24} (x-\frac{\pi}{2})^4
\end{align*}

\section{Taylor series §9.4.2}
We know from Taylor's theorem that if $f(x)$ is infinitely differentiable on some interval $|x - x_0| \leq r$, then for $n \in \naturals$, we can express $f(x)$ as
\[ f(x) = \sum_{k=0}^n \frac{1}{k!} f^{(k)}(x_0)(x-x_0)^k + (x-x_0)^{n+1} R_n(x) \]

If the remainder term satisfies
\[ (x-x_0)^{n+1} R_n(x) \rightarrow 0 \text{ as } n \rightarrow \infty \]
then $f(x)$ can be represented by the power series 
\[ f(x) = \sum_{k=0}^\infty \frac{1}{k!} f^{(k)}(x_0)(x-x_0)^k \]

This series is called the Taylor series expansion of $f(x)$ about $x=x_0$.\\
If $x_0=0$, then the Taylor series
\[ f(x) = \sum_{k=0}^\infty \frac{1}{k!} f^{(k)} (x_0) x^k \]
is known as the Maclaurin series expansion of $f(x)$.

\subsection{Remark}
Whenever $f(x)$ is infinitely differentiable, we can compute its Taylor series expansion about a point $x=x_0$ by 
\[ \sum_{k=0}^\infty \frac{1}{k!} f^{(k)} (x_0)(x-x_0)^k \]
It is then a separate question to determine if the series converges, and if it does converge, whether or not
\[ f(x) = \sum_{k=0}^\infty \frac{1}{k!} f^{(k)} (x_0)(x-x_0)^k \] 
is true.

\subsection{Example}
Compute the Taylor series for $e^x$ about $x=0$.

\subsection{Solution}
Let $f(x) = e^x$, then 
\[ f^{(k)}(x) = e^x \Rightarrow f^{(k)}(0) = 1 \]
This shows that the Taylor series for $e^x$ about $x=0$ is
\[ \sum_{k=0}^\infty \frac{1}{k!}f^{(k)}(0) x^k = \sum_{k=0}^\infty \frac{1}{k!} x^k \]

\subsection{Follow-up question}
Is it true that
\[ e^x = \sum_{k=0}^\infty \frac{1}{k!} x^k \text{?} \]

\subsection{Answer 1}
Yes, if we define
\[ e^x = \sum_{k=0}^\infty \frac{1}{k!} x^k, x \in \reals \]

\subsection{Answer 2}
Still yes even if we define $e^x$ some other way.
\[ e^x = \lim_{\lambda\rightarrow\infty} \left( 1 +\frac{x}{\lambda} \right)^\lambda \]

But you would need to show that the remainder terms $x^{n+1} R_n(x) \rightarrow 0$ as $n \rightarrow \infty$.

\subsection{Question}
Can all infinitely differentiable functions be represented by their Taylor series?

\subsection{Answer}
No. For example, define
\[ 
f(x) = \begin{cases}
e^{- \frac{1}{x}} & x>0 \\
0 & x \leq 0
\end{cases}
\]
Then it can be shown that $f(x)$ is infinitely differentiable for $x \in \reals$ but
\[ f(x) \not= \sum_{k=0}^\infty \frac{f^{(k)}(0)}{n!} x^n \text{ for } x \text{ near 0} \]

\end{document}