\documentclass[11pt]{article}

\usepackage{amsmath}
\usepackage{amsfonts}
\usepackage{amssymb}

% Give ourself extra space for text
\usepackage[left = 2.2cm, right = 2.2cm, top = 1.8cm, bottom = 2.8cm]{geometry}

% Allows us to easily change the numbering system used in things like \begin{enumerate}. https://ctan.org/tex-archive/macros/latex/contrib/enumitem/
\usepackage[shortlabels]{enumitem}

% Turns table of contents, \refs, etc. into hyperlinks
\usepackage{hyperref}

% To include image, just use \includegraphics[scale=•]{relative path to image}
\usepackage{graphicx}
\graphicspath{ {./images/} }

% Common sets
\newcommand{\integers}{\mathbb{Z}}
\newcommand{\naturals}{\mathbb{N}}
\newcommand{\reals}{\mathbb{R}}

% Power set
\newcommand{\powerset}{\mathcal{P}}

% Identity matrix
\newcommand{\ident}{\mathbb{I}}

% Inverse hyperbolic functions
\DeclareMathOperator{\arcosh}{arcosh}
\DeclareMathOperator{\arsinh}{arsinh}
\DeclareMathOperator{\artanh}{artanh}

% I hat, J hat, K hat
\newcommand{\ihat}{\boldsymbol{\hat{\textbf{\i}}}}
\newcommand{\jhat}{\boldsymbol{\hat{\textbf{\j}}}}
\newcommand{\khat}{\boldsymbol{\hat{\textbf{k}}}}

% Better vectors (for single characters)
\renewcommand{\vec}[1]{\mathbf{#1}}

% Allows us to number equations in \begin{align} statements, etc.
\newcommand\numberthis{\addtocounter{equation}{1}\tag{\theequation}}

% Augmented matrices: this allows us to make augmented matrics using something like \begin{bmatrix}[cc|c]. Taken from Stefan Kottwitz at https://tex.stackexchange.com/questions/2233/whats-the-best-way-make-an-augmented-coefficient-matrix.
\makeatletter
\renewcommand*\env@matrix[1][*\c@MaxMatrixCols c]{%
  \hskip -\arraycolsep
  \let\@ifnextchar\new@ifnextchar
  \array{#1}}
\makeatother

% NOTE: This means \section does NOT number sections, but ensures that they appear in the table of contents, which does not occur if simply \section* is used. From egreg @ https://tex.stackexchange.com/a/30225.
\setcounter{secnumdepth}{0} % sections are level 1

\begin{document}
\title{ENG1005: Lecture 28}
\author{Lex Gallon}
\maketitle

\tableofcontents

\section*{Video link}
\url{https://echo360.org.au/lesson/G_8402119b-734b-4e1e-a3b4-7e907e86ddba_b944cecf-8ba5-40d3-a870-0243a0a9e78c_2020-05-26T15:58:00.000_2020-05-26T16:53:00.000/classroom#sortDirection=desc}

\section{Initial value problems (IVP) - continued}
\subsection{Example}
Newton's law of cooling states that the temperature of a homogeneous object satisfies
\[ \frac{dT}{dt} = -K(T - T_a) \]
where $T_a$ is the ambient temperature, $T(t)$ is temperature of body at time $t$, $K > 0$ is some decay constant. The initial condition is
\[ T(t_0) = T_0 \]

Find the temperature of the body at time $t > t_0$ assuming that $K > 0$ and $T_0 > T_a$.

\subsection{Solution}
\begin{align*}
\frac{dT}{dt} = K(T - T_a) &\Rightarrow \frac{dT}{T - T_a} = -k\, dt \\
&\Rightarrow \int_{T_0}^T \frac{d\tilde{T}}{T-T_a} = -K \int_{t_0}^{t} d\tilde{t} \\
&\Rightarrow \left. \ln(\tilde{T} - T_a) \right|_{T_0}^T = -K t|_{t_0}^t \\
&\Rightarrow \ln\left( \frac{T-T_a}{T_0 - T_a} \right) = -K(t - t_0) \\
&\Rightarrow \frac{T-T_a}{T_0 - T_a} = e^{-K(t - t_0)} \\
&\Rightarrow T = (T_0 - T_a) e^{-K(t - t_0)} + T_a
\end{align*}
Note that 
\[ \lim_{t \rightarrow \infty} T(t) = T_a \]

\subsection{Remark}
The IVP for an $n$th order ODE
\[ \frac{d^ny}{dt^n} = F \left( x, y, \frac{dy}{dt}, \frac{d^2y}{dt^2}, ..., \frac{d^{n-1}y}{dt^{n-1}} \right) \]
need $n$ initial conditions
\[ y(t_0) = y_0, \frac{dy}{dt}(t_0)=y_1, ... \frac{d^{n-1}y}{dt^{n-1}}(t_0) = y_{n-1} \]
to be \textbf{uniquely solvable}.

\section{Separable ODEs §10.5.3}
A 1st order ODE
\[ \frac{dy}{dy} = F(t, y) \]
is called separable if $F(t,y)$ can be written as
\[ F(t, y) = \frac{G(t)}{H(y)} \]
To solve these equations, we write
\[ \frac{dy}{dt} = \frac{G(t)}{H(y)} \]
as
\[ H(y) \frac{dy}{dt} = G(t) \]
Then we integrate both sides with respect to $t$ to get
\[ \int H(y) \frac{dy}{dt}\, dt = \int G(t)\, dt \]
We then use change of variables
\[ y = y(t),\quad dy = \frac{dy}{dt}\, dt \]
to get
\[ \int H(y)\, dy = \int G(t)\, dt \]
In general, this yields an implicit solution to the above ODE.

For IVP with initial condition
\[ y(t_0) = y_0, \]
then we can write the solution as
\[ \int_{y_0}^y H(\tilde{y})\, d\tilde{t} = \int_{t_0}^t G(\tilde{t})\, d\tilde{t} \]

\subsection{Informal version}
\begin{align*}
\frac{dy}{dt} = \frac{G(t)}{h(y)} &\Rightarrow H(y)\, dy = G(t) \, dt \\
&\Rightarrow \int H(y)\, dy = \int G(t) \, dt
\end{align*}

\subsection{Example}
\[ \frac{dV}{dt}= \sqrt{V} \] 

\subsection{Solution}
This is separable since we can trivially write the RHS as
\[ \sqrt{V} = \frac{1}{\frac{1}{\sqrt{V}}} = \frac{G(t)}{H(V)} \]

\begin{align*}
\frac{dV}{dt} = \sqrt{V} &\Rightarrow \frac{1}{\sqrt{V}}\, dV = dt \\
&\Rightarrow \int \frac{1}{\sqrt{V}}\, dV = \int dt \\
&\Rightarrow 2\sqrt{V} = t + 2c \quad (\text{only do } 2c \text{ to simplify next step}) \\
&\Rightarrow \sqrt{V} = \frac{t}{2} + c \\
&\Rightarrow V = \left( \frac{t}{2} + c \right)^2
\end{align*}

\section{Linear 1st order ODEs §10.5.9}
The most general 1st order ODE is
\[ \frac{dy}{dt} + p(t)y = q(t) \]
If $q(t) = 0$, then the ODE is said to be \textbf{homogeneous}, and otherwise it is called inhomogeneous or nonhomogeneous.\\
To solve this equation, we multiply it by an integrating factor $I(t)$ to get 
\[ I \frac{dy}{dt} + Ip(t)y = Iq(t) \]
or equivalently
\[ \frac{d}{dt}(Iy) - \frac{dI}{dt}y + Ipy = Iq \]
We can write the above ODE as
\[ \frac{d}{dt}(Iy) + \left( Ip - \frac{dI}{dt} \right)y = Iq \]
Now, we could simplify this is we choose $I$ such that those terms in the right set of brackets become zero. So we choose $I$ to satisfy
\[ \frac{dI}{dt} = Ip \]
But this is a separable ODE! So,
\begin{align*}
\frac{dI}{dt} = Ip &\Rightarrow \frac{dI}{I} = p\, dt \\
&\Rightarrow \int \frac{dI}{I} = \int p\, dt \\
&\Rightarrow \ln(I) = \int p(t)\, dt \\
&\Rightarrow I(t) = e^{\int p(t)\, dt}
\end{align*}
With this choice of $I(t)$, we get that 
\[ \frac{d}{dt}(Iy) = Iq. \]
Integrate both sides to get
\begin{align*}
I(t) y(t) &= \int I(t) q(t)\, dt + c \\
\Rightarrow y(t) &= \frac{1}{I(t)} \left[ \int I(t) q(t)\, dt + c \right],
\end{align*}
where
\[ I(t) = e^{\int p(t)\, dt}. \]

\subsection{Example}
Solve the IVP
\[ \frac{dy}{dt} + ty = t \]
\[ y(0) = 2.\]

\subsection{Solution}
In our case,
\[ p(t) = t,\ q(t) = t \]
So we can find the integrating factor
\[ I(t) = e^{\int t\, dt} = e^{\frac{1}{2} t^2} \quad (\text{note no need to worry about integration constant}). \]
So the general solution is given by
\begin{align*}
y(t) &= \frac{1}{I(t)} \left[ \int I(t) q(t)\, dt + c \right] \\
&= e^{-\frac{1}{2} t^2} \left[ \int e^{\frac{1}{2} t^2} t\, dt + c \right] \\
&= e^{-\frac{1}{2} t^2} \left[ e^{\frac{1}{2} t^2} + c \right] \\
&= 1 + c e^{-\frac{1}{2} t^2}.
\end{align*}
And now, for the initial condition,
\begin{align*}
y(0) = 2 &\Rightarrow y(0) 1 + c e^{-\frac{1}{2} t^2} = 1 + c = 2 \\
&\Rightarrow c = 1.
\end{align*}
So we can conclude that
\[ y(t) = 1 + e^{-\frac{1}{2} t^2} \]
solves the IVP.

\section{Linear differential equations §10.8}
\subsection{Superposition principle}
Suppose $y_1(x)$ and $y_2(x)$ both solve the following linear homogeneous ODE
\[ \sum_{k=0}^n a_k(x) \frac{d^ky}{dx^k} = 0 \]
Let
\[ \tilde{y}(x) = \lambda_1 y_1(x) + \lambda_2 y_2(x),\ \lambda_1, \lambda_2 \in \reals \]
Then
\begin{align*}
\sum_{k=0}^n a_k(x) \frac{d^ky}{dx^k} &= \sum_{k=0}^n a_k(x) \left( \lambda_1 \frac{d^k y_1}{dx^k} + \lambda_2 \frac{d^k y_2}{dx^k} \right) \\
&= \lambda_1 \sum_{k=0}^n a_k(x) \frac{d^k y}{dx^k} + \lambda_2 \sum_{k=0}^n a_k(x) \frac{d^k y}{dx^k} \\
&= 0
\end{align*}
shows that
\[ \tilde{y}(x) = \lambda_1 y_1(x) + \lambda_2 y_2(x),\ \lambda_1, \lambda_2 \in \reals \]
solves the linear, homogeneous ODE.

\end{document}