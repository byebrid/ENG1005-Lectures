%%%%% This is a template for notes for ENG1005 lectures %%%%%

\documentclass[11pt]{article}

\usepackage{amsmath}
\usepackage{amsfonts}
\usepackage{amssymb}

% Turns table of contents, \refs, etc. into hyperlinks
\usepackage{hyperref}

% Common sets
\newcommand{\integers}{\mathbb{Z}}
\newcommand{\naturals}{\mathbb{N}}
\newcommand{\reals}{\mathbb{R}}

% Inverse hyperbolic functions
\DeclareMathOperator{\arcosh}{arcosh}
\DeclareMathOperator{\arsinh}{arsinh}
\DeclareMathOperator{\artanh}{artanh}

% Allows us to number equations in \begin{align} statements, etc.
\newcommand\numberthis{\addtocounter{equation}{1}\tag{\theequation}}

% NOTE: This means \section does NOT number sections, but ensures that they appear in the table of contents, which does not occur if simply \section* is used. From egreg @ https://tex.stackexchange.com/a/30225.
\setcounter{secnumdepth}{0} % sections are level 1

\begin{document}
\title{ENG1005: Lecture 4}
\author{Lex Gallon}
\maketitle

\tableofcontents

\section{Improper integrals: infinite integrals §9.2.2}
Define integrals like
\[ \int_0^\infty \frac{1}{1+x^2} \,dx \]

We first regularize the integral by
\[ \int_0^\varepsilon  \frac{1}{1+x^2} \,dx,\  \varepsilon>0 \]

Then we define the integral as a limit
\[ \int_0^\infty \frac{1}{1+x^2} \,dx = \lim_{\varepsilon\rightarrow\infty } \left( \int_0^\varepsilon  \frac{1}{1+x^2} \,dx \right) = \lim_{\lambda\searrow 0} \left( \int_0^{\frac{1}{\lambda}}  \frac{1}{1+x^2} \,dx \right) \]

So
\begin{align*}
\int_0^\infty \frac{1}{1+x^2} \,dx &= \lim_{\varepsilon\rightarrow\infty } \left( \int_0^\varepsilon  \frac{1}{1+x^2} \,dx \right) \\
 &=  \lim_{\varepsilon\rightarrow\infty } \left( \arctan(\varepsilon) - \arctan(0) \right) \\
 &=  \lim_{\varepsilon\rightarrow\infty } \left( \arctan(\varepsilon) \right) \\
 &= \frac{\pi}{2} 
\end{align*}

This shows that $\int_0^\infty \frac{1}{1+x^2} \,dx$ is a convergent improper integral.

<INSERT PICTURE HERE!>

\section{Summary}
\begin{enumerate}
\item If $f(x)$ is continuous on $[a, \infty)$, then
\[ \int_a^\infty f(x) \,dx = \lim_{\varepsilon\rightarrow\infty} \int_a^\varepsilon f(x) \,dx \]

\item If $f(x)$ is continuous on $(-\infty, \infty)$, then
\[ \int_{-\infty}^\infty f(x) \,dx = \lim_{\varepsilon\rightarrow -\infty} \int_\varepsilon^c f(x) \,dx + \lim_{\lambda\rightarrow \infty} \int_c^\lambda f(x) \,dx  \]
\end{enumerate}

<MAYBE INSERT PICTURE HERE>

\section{Comparison principle: convergence}
\subsection*{Definition}
\[ \int_1^\infty e^{-x^2} \,dx = \lim_{\varepsilon\rightarrow\infty} \int_1^\varepsilon e^{-x^2} \,dx \]
but we can't compute $\int_1^\varepsilon e^{-x^2} \,dx$.

\subsection*{Theorem (Comparison Test)}
Suppose $f(x)$ and $g(x)$ (not magnitude of g(x)?) are continuous on $[a, \infty)$, $|f(x)| \leq g(x) $ for all $ x\in[a,\infty) $, and $\int_a^\infty g(x) \,dx$ converges.

\[ \text{Then } \int_a^\infty f(x) \,dx \text{ also converges} \]

\subsection*{Proof}
If $|f(x)| \leq g(x) $, then
\[ \left| \int_a^\varepsilon f(x) \,dx \right| \leq \int_a^\varepsilon |f(x)| \,dx \leq \int_a^\varepsilon g(x) \,dx \]
(assuming g(x) converges)

\subsection*{Example}
Since
\[ \left| e^{-x^2} \right| = e^{-x^2} \leq e^{-x} \]
(couldn't we just say it's less than or equal to 1? No, because when we take the limit to infinity, it wouldn't converge!).

and
\begin{align*}
\int_1^\infty e^{-x} \,dx &= \lim_{\varepsilon\rightarrow\infty} \int_1^\varepsilon e^{-x} \,dx \\
 &= \lim_{\varepsilon\rightarrow\infty} \left( -e^{-\varepsilon} + e^{-1} \right)
\end{align*}

So by the Comparison Test, we conclude that 
\[ \int_1^\infty e^{-x} \,dx \text{ is a convergent improper integral} \]

\subsection*{Example}
Determine if the improper integral
\[ \int_1^\infty \frac{1}{1+xe^x} \,dx \]
converges or diverges.

\subsection*{Solution}
\begin{align*}
1 &\leq x \\
e^x &\leq xe^x \\
e^x &\leq 1 + xe^x \\
\frac{1}{1 + xe^x} &\leq \frac{1}{e^x} \\
\left| \frac{1}{1 + xe^x} \right| &\leq e^{-x}
\end{align*}

Since
\[ \text{Since } \int_1^\infty e^{-x} \,dx \text{ is convergent, we deduce from the Comparison test that } \int_1^\infty \frac{1}{1 + xe^x} \,dx \text{ is also convergent.}\]

\subsection*{Theorem (Comparsion Test)}
Suppose $f(x)$ and $g(x)$ are continuous on $[a, b)$, $|f(x)| \leq g(x) $ for all $ x\in[a,b) $, and $\int_a^b g(x) \,dx$ converges.

Then 
\[ \int_a^b f(x) \,dx \text{ converges.}\]

\subsection*{Example}
\[ \int_0^\frac{\pi}{4} \frac{\cos(x)}{\sqrt{x}} \,dx \text{ convergent or not?} \]

\subsection*{Solution}
\[ 
\left| \frac{\cos(x)}{\sqrt(x)} \right| = \frac{1}{\sqrt{x}} |\cos(x)| \leq \frac{1}{\sqrt{x}}
\]

\begin{align*}
\int_0^\frac{\pi}{4} \frac{\cos(x)}{\sqrt{x}} \,dx &= \lim_{\varepsilon\searrow 0} \int_\varepsilon^\frac{\pi}{4}  \frac{1}{\sqrt{x}} \,dx \\
 &= \lim_{\varepsilon\searrow 0} 2\sqrt{\frac{\pi}{4}} - 2\sqrt{\varepsilon} \\
 &= \sqrt{\pi}
\end{align*}




\end{document}