\documentclass[11pt]{article}

\usepackage{amsmath}
\usepackage{amsfonts}
\usepackage{amssymb}

% Give ourself extra space for text
\usepackage[left = 2.2cm, right = 2.2cm, top = 1.8cm, bottom = 2.8cm]{geometry}

% Allows us to easily change the numbering system used in things like \begin{enumerate}. https://ctan.org/tex-archive/macros/latex/contrib/enumitem/
\usepackage[shortlabels]{enumitem}

% Turns table of contents, \refs, etc. into hyperlinks
\usepackage{hyperref}

% Common sets
\newcommand{\integers}{\mathbb{Z}}
\newcommand{\naturals}{\mathbb{N}}
\newcommand{\reals}{\mathbb{R}}

% Inverse hyperbolic functions
\DeclareMathOperator{\arcosh}{arcosh}
\DeclareMathOperator{\arsinh}{arsinh}
\DeclareMathOperator{\artanh}{artanh}

% I hat, J hat, K hat
\newcommand{\ihat}{\boldsymbol{\hat{\textbf{\i}}}}
\newcommand{\jhat}{\boldsymbol{\hat{\textbf{\j}}}}
\newcommand{\khat}{\boldsymbol{\hat{\textbf{k}}}}

% Better vectors (for single characters)
\renewcommand{\vec}[1]{\mathbf{#1}}

% Allows us to number equations in \begin{align} statements, etc.
\newcommand\numberthis{\addtocounter{equation}{1}\tag{\theequation}}

% Augmented matrices: this allows us to make augmented matrics using something like \begin{bmatrix}[cc|c]. Taken from Stefan Kottwitz at https://tex.stackexchange.com/questions/2233/whats-the-best-way-make-an-augmented-coefficient-matrix.
\makeatletter
\renewcommand*\env@matrix[1][*\c@MaxMatrixCols c]{%
  \hskip -\arraycolsep
  \let\@ifnextchar\new@ifnextchar
  \array{#1}}
\makeatother

% NOTE: This means \section does NOT number sections, but ensures that they appear in the table of contents, which does not occur if simply \section* is used. From egreg @ https://tex.stackexchange.com/a/30225.
\setcounter{secnumdepth}{0} % sections are level 1

\begin{document}
\title{ENG1005: Lecture 15}
\author{Lex Gallon}
\maketitle

\tableofcontents

\section*{Video link}
Click \href{https://echo360.org.au/lesson/G_35fe23e0-41ee-4e6f-b0f5-05f4155bb7b0_b944cecf-8ba5-40d3-a870-0243a0a9e78c_2020-04-23T15:58:00.000_2020-04-23T16:53:00.000/classroom#sortDirection=desc}{here} for a recording of the lecture.

\section{Matrix operations continued}
\subsection{Transposition}
\[ [A_{ij}]^T = [A_{ji}] \]
This means the transposition of an $m \times n$ matrix is an $n \times m$ matrix.

\subsection{Example}
\[
\begin{bmatrix}
0 & 1 & 2 \\
-3 & 4 & 6
\end{bmatrix}
= \begin{bmatrix}
0 & -3 \\
1 & 4 \\
2 & 6
\end{bmatrix}
\]

\section{Properties of matrix operations §5.2.6}
\begin{enumerate}[ (i) ]
\item $(AB)C = A(BC)$ (associative)
\item $(A + B)C = AC + BC$ (distributive)\\
This also works with scalars: $\lambda(B + C) = \lambda B + \lambda C$

\subsection{Remark}
In general, matrix multiplication is not commutative,
\[ AB \not = BA \]

\subsection{Example}
\[
\begin{bmatrix}
0 & 1 \\
1 & 0
\end{bmatrix}
\begin{bmatrix}
1 & 0 \\
0 & -1
\end{bmatrix}
= \begin{bmatrix}
0 & -1 \\
1 & 0
\end{bmatrix}
\]
But if you switch the matrics around, you instead get
\[
\begin{bmatrix}
1 & 0 \\
0 & -1
\end{bmatrix}
\begin{bmatrix}
0 & 1 \\
1 & 0
\end{bmatrix}
= \begin{bmatrix}
0 & 1 \\
-1 & 0
\end{bmatrix}
\]
So you can see that the order in which you multiply matrices does matter.

\item $(AB)^T = B^T A^T$

\item The matrix
\[ \mathbb{I}_n = (\delta_{ij}),\ 1 \leq i, j \leq n \]
Note: $\delta_{ij}$ is the Kronecker delta and is defined as
\[ \delta_{ij} = \begin{cases}
  1 \text{ if } i = j, \\
  0 \text{ if } i \not = j
\end{cases}
\]

\subsection{Example}
\[ \mathbb{I}_3 = \begin{bmatrix}
1 & 0 & 0 \\
0 & 1 & 0 \\
0 & 0 & 1
\end{bmatrix}
\]
\end{enumerate}

If $A$ is an $m \times n$ matrix and $B$ is an $n \times p$ matrix, then
\[ A \mathbb{I}_n = A \]
\[ \mathbb{I}_n B = B \]

If $C$ is an $n \times n$ matrix, then
\[ \mathbb{I}_n C = C \mathbb{I}_n = C \]

\section{Matrix formulation of linear equations §5.5}
\begin{align*}
A_{11}x_1 + A_{12}x_2  + ... + A_{1n}x_n &= b_1 \\
A_{21}x_1 + A_{22}x_2  + ... + A_{2n}x_n &= b_2 \\
\vdots \\
A_{m1}x_1 + A_{m2}x_2  + ... + A_{mn}x_n &= b_m \\
\end{align*}

We can write the above system as
\[ A \vec{x} = \vec{b} \]
where
\[ 
A = \begin{bmatrix}
A_{11} & A_{12} & ... & A_{1n} \\
A_{21} & A_{22} & ... & A_{2n} \\
\vdots & \vdots & \ddots & \vdots \\
A_{m1} & A_{m2} & ... & A_{mn} \\
\end{bmatrix}
\]
\[ 
\vec{x} = \begin{bmatrix}
x_1 \\
x_2 \\
\vdots \\
x_n
\end{bmatrix}
\text{ and } \vec{b} = \begin{bmatrix}
b_1 \\
b_2 \\
\vdots \\
b_n
\end{bmatrix}
\]

\subsection{Terminology}
Matrices of the form $\displaystyle{ 
x = \begin{bmatrix}
x_1 \\
x_2 \\
\vdots \\
x_n
\end{bmatrix} }$ and $y = \begin{bmatrix}
y_1 & y_2 & ... & y_n
\end{bmatrix}$ are referred to as column and row vectors, respectively, of length $n$.

\section{Rank §5.6}
Given an $m \times n$ matrix $A$, we can perform Gaussian elimination to put it into row echelon form $B$. Note that, although there is no unique row echelon form, the number of pivots will always be the same. Then the rank of $A$ is defined by
\[ \text{rank}(A) = \# \text{ of pivots of } A \]

\subsection{Example}
We previously showed using Gaussian elimination that the matrix
\[
A = \begin{bmatrix}
1 & 3 & -1 & 1 \\
2 & 4 & 2 & 1 \\
-3 & -9 & 3 & -3 
\end{bmatrix}
\]
can be row-reduced to
\[
\begin{bmatrix}
1 & 3 & -1 & 1 \\
0 & 1 & -2 & \frac{1}{2} \\
0 & 0 & 0 & 0
\end{bmatrix}
\]

Since there are 2 pivots, we get
\[ \text{rank}(A) = 2 \]

\section{The solutions space for linear systems of equations}
\subsection{Theorem}
Suppose $A$ is an $m \times n$ matrix and $\vec{b}$ is an $m$-column vector. Then the matrix equation
\[ A \vec{x} = \vec{b} \]
has
\begin{enumerate}[ (i) ]
\item no solutions if $\text{rank}(A) \not = \text{rank}([A | \vec{b}] )$,
\item and at least one non-trivial solution ($\vec{x} \not = \vec{0}$, presumably also $\vec{b} \not = \vec{0}$) if $\text{rank}(A) = \text{rank}([A | \vec{b}] )$.

Furthermore, if $\text{rank}(A) = \text{rank}([A | \vec{b}] )$, then\\
\begin{enumerate}[label=(\roman{enumi}.\alph*)]
\item there is a unique solution if $\text{rank}(A) = n$,

\item and an infinite number of solutions with $n$-rank($A$) free parameters if $\text{rank}(A) < n$.
\end{enumerate}

\end{enumerate}

\section{LU decomposition}
Gaussian elimination is an effective method for solving systems of linear equations
\[ A \vec{x} = \vec{b} \quad (*) \]
\begin{itemize}
\item For $n \times n$ matrices $A$, the computational effort to solve (*) is $n^3$.
\item $n = 10^6$ is not considered large nowadays!
\end{itemize}

Gaussian elimination acts on the augmented matrix. Often you may have a constant $A$, but then have many different vectors $\vec{b}$ so Gaussian elimination can be inefficient in this case since it requires $A$ gets changed everytime $\vec{b}$ is.

\subsection{Definition}
A factorisation of an $n \times n$ matrix of the form
\[ A = LU \]
where $L$ is a lower triangular matrix, and $U$ is an upper triangular matrix (both are also $n \times n$) is known as an LU-decomposition for $A$.

\[ 
\begin{bmatrix}
1 & 0 & 0 \\
3 & -1 & 0 \\
4 & 9 & 2
\end{bmatrix}
\quad
\begin{bmatrix}
7 & -2 & -1 \\
0 & 9 & 6 \\
0 & 0 & 4 \\
\end{bmatrix}
\]
The matrix on the left is a lower triangular matrix, and the one on the right is an upper triangular matrix.

To solve
\[ A \vec{x} = \vec{b} \]
we use LU decomposition to write it as follows
\[ LU \vec{x} = \vec{b} \quad (**) \]

To solve (**), we do the following:
\begin{enumerate}[ (i) ]
\item Set $\vec{y} = U \vec{x}$.\\
Then (x) becomes
\[ L \vec{y} = \vec{b} \quad (**) \]
which can solve by back-substitution to get $\vec{y}$(since $L$ is in row echelon form).

\item We then solve $U \vec{x} = \vec{y}$\\
using forward substitution to get $\vec{x}$
\end{enumerate}
\end{document}