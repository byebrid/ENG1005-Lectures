\documentclass[11pt]{article}

\usepackage{amsmath}
\usepackage{amsfonts}
\usepackage{amssymb}

% Give ourself extra space for text
\usepackage[left = 2.2cm, right = 2.2cm, top = 1.8cm, bottom = 2.8cm]{geometry}

% Allows us to easily change the numbering system used in things like \begin{enumerate}. https://ctan.org/tex-archive/macros/latex/contrib/enumitem/
\usepackage[shortlabels]{enumitem}

% Turns table of contents, \refs, etc. into hyperlinks
\usepackage{hyperref}

% Common sets
\newcommand{\integers}{\mathbb{Z}}
\newcommand{\naturals}{\mathbb{N}}
\newcommand{\reals}{\mathbb{R}}

% Inverse hyperbolic functions
\DeclareMathOperator{\arcosh}{arcosh}
\DeclareMathOperator{\arsinh}{arsinh}
\DeclareMathOperator{\artanh}{artanh}

% Allows us to number equations in \begin{align} statements, etc.
\newcommand\numberthis{\addtocounter{equation}{1}\tag{\theequation}}

% NOTE: This means \section does NOT number sections, but ensures that they appear in the table of contents, which does not occur if simply \section* is used. From egreg @ https://tex.stackexchange.com/a/30225.
\setcounter{secnumdepth}{0} % sections are level 1

\begin{document}
\title{ENG1005: Lecture 10}
\author{Lex Gallon}
\maketitle

\tableofcontents

\section*{Video link}
Click \href{https://echo360.org.au/lesson/G_8402119b-734b-4e1e-a3b4-7e907e86ddba_b944cecf-8ba5-40d3-a870-0243a0a9e78c_2020-04-07T15:58:00.000_2020-04-07T16:53:00.000/classroom#sortDirection=desc}{here} for a recording of the lecture.

\section{Taylor series}
\subsection{Example}
Find the Taylor series for $\ln(x)$ about $x=1$ and determine the interval of convergence.

\subsection{Solution}
\[ f(x) = \sum_{k=0}^\infty \frac{1}{k!} f^{(k)}(x_0)(x-x_0)^k , f(x) = \ln(x) \]

\[ \frac{d}{dx}\ln(x) = \frac{1}{x},  \frac{d^2}{dx^2}\ln(x) = \frac{-1}{x^2}, \frac{d^3}{dx^3}\ln(x) = \frac{2}{x^3}, \frac{d^4}{dx^4}\ln(x) = \frac{-6}{x^4} \]

\[ \Rightarrow \frac{d^k}{dx^k}\ln(x) =(-1)^{k-1} \frac{(k-1)!}{x^k} ,\ k=1,2,3,... \]

\[ \ln(1) = 0,\ \left. \frac{d^k}{dx^k}\ln(x) \right|_{x=1} = (-1)^{k-1} (k-1)! \]

So the Taylor series for $\ln(x)$ about $x=1$ is
\begin{align*}
\sum_{k=1}^\infty \frac{1}{k!} (-1)^{k-1} (k-1)!(x-1)^k &= \sum_{k=1}^\infty \frac{(-1)^{k-1}}{k} (x-1)^k
\end{align*}
%*& (\text{note k goes from 1 since we know \ln(

The radius of convergence is given by
\[
r = \lim_{k\rightarrow\infty} \left| \frac{\frac{(-1)^{k-1}}{k}}{\frac{(-1)^k}{k+1}} \right| *= \lim_{k\rightarrow\infty} \frac{k+1}{k} = \lim_{k\rightarrow\infty} \left( 1 + \frac{1}{k} \right) = 1
\]

We conclude that the series converges uniformly for
\[ |x-1| < 1 \Leftrightarrow 0 < x < 2 \]

\subsection{Fact}
\[ \ln(x) = \sum_{n=1}^\infty \frac{(-1)^{n-1)}}{n} (x-1)^n, 0 < x \leq 2 \]

What happens for x > 2? Try calculating the Taylor series for $\ln(x)$ about $x=3$.

\section{L'Hôpital's rule §9.4.3}
Suppose $f(x)$ and $g(x)$ satisfy
\begin{enumerate}[ (i) ]
\item $f''(x)$ and $g''(x)$ are continuous on $|x-x_0| < r$
\item $f(x_0)=g(x_0)=0$ but $g'(x_0) \not= 0$.
\end{enumerate}

Then
\[ \lim_{x\rightarrow x_0} \frac{f(x)}{g(x)} =  \lim_{x\rightarrow x_0} \frac{ f(x_0) + f'(x_0)(x-x_0) + (x-x_0)^2 R_2^f(x) }{ g(x_0) + g'(x_0)(x-x_0) + (x-x_0)^2 R_2^g(x) }   \]
We know $f(x_0)=g(x_0)=0$ and $g'(x_0) \not= 0$ so dividing both numerator and denominator by $(x-x_0)$ gives us
\[ \lim_{x\rightarrow x_0} \frac{f(x)}{g(x)} = \lim_{x\rightarrow x_0} \frac{f'(x_0)(x-x_0) + (x-x_0) R_2^f(x) }{ g'(x_0) + (x-x_0) R_2^g(x) } \]
And since $\lim_{x\rightarrow x_0} (x-x_0) = 0$, we get
\[ \lim_{x\rightarrow x_0} \frac{f(x)}{g(x)} = \frac{f'(x_0)}{g'(x_0)} = \lim_{x\rightarrow x_0} \frac{f'(x)}{g'(x)} \]

\subsection{Example}
Compute 
\[ \lim_{x\rightarrow 0} \frac{\sin(x)}{x} \]
% Note it's '0÷0'

\subsection{Solution}
\[ \lim_{x\rightarrow 0 } \frac{\sin(x)}{x} = \lim_{x	\rightarrow 0 } \frac{\sin '(x))}{x'} = \lim_{x\rightarrow0} \frac{\cos(x)}{1} = 1 \]

\subsection{Example}
What is $\displaystyle{\lim_{x\rightarrow 0 } \frac{e^x-1}{x}}$?

\subsection{Solution}
\[ \lim_{x\rightarrow 0 } \frac{e^x-1}{x} = \lim_{x\rightarrow 0 } \frac{\frac{d}{dx} \left( e^x-1 \right)}{\frac{d}{dx}(x)} \lim_{x\rightarrow 0 } \frac{e^x}{1} = 1 \]

\subsection{Remark}
L'Hôpital's rule $\displaystyle{\lim_{x\rightarrow x_0 } \frac{f(x)}{g(x)} = \lim_{x\rightarrow x_0 } \frac{f'(x)}{g'(x)}}$ remains valid for:
\begin{enumerate}[ (i) ]
\item one sided limits, $x \rightarrow x_0^\pm$
\item infinite limits, $x \rightarrow \pm \infty$
\item $\displaystyle{\frac{\infty}{\infty}}$ limits, i.e. $\displaystyle{\lim_{x \rightarrow x_0}} f(x) = \pm \infty$
\end{enumerate}

\subsection{Example}
Compute
\[ \lim_{x \searrow 0} x \ln(x) \]

\subsection{Solution}
\begin{align*}
\lim_{x \searrow 0} x \ln(x) &= \lim_{x \searrow 0} \frac{\ln(x)}{\frac{1}{x}} & (\frac{\infty}{\infty}) \\
&= \lim_{x \searrow 0} \frac{\ln'(x)}{(\frac{1}{x})'} \\
&= \lim_{x \searrow 0} \frac{\frac{1}{x}}{\frac{-1}{x^2}} \\
&= \lim_{x \searrow 0} (-x) \\
&= 0
\end{align*}

\section{Vectors §4.2.1 - 4.2.8}
\subsection[Vectors in R3]{Vectors in $\reals^3$}
$\underline{u} = (u_1, u_2, u_3)$, ($\vec{u} = (u_1, u_2, u_3)$)

\subsection{Standard basis}
$\vec{i} = \underline{e}_1 = (1, 0, 0)$
$\vec{j} = \underline{e}_2 = (1, 0, 0)$
$\vec{k} = \underline{e}_3 = (1, 0, 0)$

\subsection{n-dimensions}
\begin{align*}
\underline{u} &= (u_1, u_2, u_3, ..., u_n) \\
\underline{e}_i &= (0, 0, ..., 0, 1, 0, ..., 0) & (\text{1 at } i^{\text{th}} \text{ position}) \\
\underline{u} &= \sum_{i=1}^n u_i\,\underline{e}_i
\end{align*}

\subsection{Expansion in the standard basis}
\begin{align*}
\underline{u} &= (u_1, u_2, u_3) \\
&= u_1(1, 0, 0) + u_2(0, 1, 0) + u_3(0, 0, 1) \\
&= u_1 \underline{i} + u_2\underline{j} + u_3\underline{k} \\
\end{align*}


\end{document}