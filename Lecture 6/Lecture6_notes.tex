\documentclass[11pt]{article}

\usepackage{amsmath}
\usepackage{amsfonts}
\usepackage{amssymb}

\usepackage{hyperref}

\newcommand{\integers}{\mathbb{Z}}
\newcommand{\naturals}{\mathbb{N}}
\newcommand{\reals}{\mathbb{R}}

% NOTE: This means \section does NOT number sections, but ensures that they appear in the table of contents, which does not occur if simply \section* is used. From egreg @ https://tex.stackexchange.com/a/30225.
\setcounter{secnumdepth}{0} % sections are level 1


\begin{document}
\title{Lecture 6 notes}
\author{Lex Gallon}
\maketitle

\tableofcontents

\section{Sequences §7.2}
\subsection*{Definition}
A sequence is a countably infinite collection of numbers
\[ \{a_n\}_{n=1}^\infty = \{ a_1, a_2, a_3, ...\} \]

\subsection*{Example}
\[ a_n = \frac{1}{n}, n\in \naturals \]

\section{Limits of sequences §7.5, 7.5.1}
A sequence $\{a_n\}_{n=1}^\infty$ is said to converge to a limit $l$, denoted
if for any $\varepsilon>0$, there exists $N=N(\varepsilon), \varepsilon\in \naturals$ such that $|a_n-l| < \varepsilon|$ for $n>N$.

\subsection*{Example}

Show that 
\[ \lim_{n\rightarrow\infty} \frac{1}{n} = 0 \]

\subsection*{Solution}

Fix $\varepsilon > 0$, choose $N \in \naturals$ such that $N>\frac{1}{\varepsilon}$.

Then
\[ n\geq N \Rightarrow \frac{1}{n} \leq \frac{1}{N} < \varepsilon \]

and so
\[ |\frac{1}{n} - 0| = \frac{1}{n} = \frac{1}{n} < \varepsilon \]

\section{Properties of limits §7.5.2}
\subsection*{Theorem}
Suppose $\lim_{n\rightarrow\infty} a_n = l$ and $\lim_{n\rightarrow\infty} b_n = m$, then
\begin{enumerate}
\item \[ \lim_{n\rightarrow\infty} \left( a_n + b_n \right) = \lim_{n\rightarrow\infty} a_n + \lim_{n\rightarrow\infty} b_n\]

\item \[ \lim_{n\rightarrow\infty} \left( a_n b_n \right) = \lim_{n\rightarrow\infty} a_n \lim_{n\rightarrow\infty} b_n\]

\item \[ \lim_{n\rightarrow\infty} \left( \frac{a_n}{b_n} \right) = \frac{\lim_{n\rightarrow\infty} a_n}{\lim_{n\rightarrow\infty} b_n} \text{, provided that } m\neq0 \] 
\end{enumerate}

\subsection*{Theorem (Uniqueness of limit}
Suppose $\lim_{n\rightarrow\infty} a_n = l$ and $\lim_{n\rightarrow\infty} a_n = l'$. Then $l=l'$. Moreover, if $\{a_{n_j}\}_{j=1}^\infty$ is my sub-sequence of $\{a_n\}_{n=1}^\infty$. 
%For example, $n_j$ could equal $2n$, which would lead to getting every second element in $a_n$.
Then
\[ lim_{n\rightarrow\infty} \left( a_{n_j} \right) = lim_{n\rightarrow\infty} \left( a_n \right) =l \]

For example, if you took the limit of every second term in the sequence, it would still be the same as the limit of every term.

\subsection*{Theorem}
Suppose $\lim_{x\rightarrow x_0}f(x)=l$ and $\lim_{n\rightarrow\infty}a_n=x_0$. Then
\[ \lim_{n\rightarrow\infty}f(a_n)=l \]

Conversely, if $\lim_{n\rightarrow\infty}f(a_n) = l$ for all sequences $\{a_n\}_{n=1}^\infty$ satisfying $\lim_{n\rightarrow\infty}a_n=x_0$, then $\lim_{x\rightarrow x_0}f(x)=l$.

\section{Non-existence of limits}
There can be many reasons why limits of sequences do not exist. The two most common are:

\begin{enumerate}
\item Oscillation

The limit $\lim_{n\rightarrow\infty}(-1)^n$ DNE.

\item Unboundedness

The limit $\lim_{n'\rightarrow\infty}ln(n)$ DNE.
\end{enumerate}

\section{Series §7.6}
\subsection*{Definition}
A sum of the elements of a sequence $\{ a_n \}_{n=1}^\infty$,

i.e.
\[ \sum_{n=1}^\infty a_n = a_1 + a_2 + a_3 + ... \]
is called a series.

\subsection*{Examples}
\[ \sum_{n=1}^\infty \frac{1}{n} = 1 + \frac{1}{2} + \frac{1}{3} + ... \]
\[ \sum_{n=1}^\infty \frac{1}{n^2} = 1 + \frac{1}{4} + \frac{1}{9} + ... \]
\[ \sum_{n=0}^\infty (-1)^n = 1 + (-1) + 1 + (-1) + ... \]

\subsection*{Definition}
Given a sequence $\{ a_n \}_{n=1}^\infty$ and $N\in\naturals$, the Nth partial sum is defined by

\[ S_N = \sum_{n=1}^N a_n = a_1 + a_2 + a_3 + ... + a_N \]

\section{Convergence and divergence of a series}
\subsection*{Definition}
A series $ \sum_{n=1}^\infty a_n $ is said to converge (diverge) if the limit $\lim_{N\rightarrow\infty}S_N$ of partial sums exists (does not exist).

If $\lim_{N\rightarrow\infty}S_N$ converges, then we define

\[ \sum_{n=1}^\infty a_n := \lim_{N\rightarrow\infty}S_N \]

\section{Geometric Series §7.3.2}
\subsection*{Definition}
\[ \sum_{n=0}^\infty a^n = 1 + a + a^2 + ... \]
Note: this could be multiplied by a constant and still be a geometric series.

Consider the partial sums
\[ S_N =  \sum_{n=0}^N a^n = 1 + a + a^2 + ... + a^N \]

Multiply by a to get
\[ aS_N =  \sum_{n=0}^N a^n = a + a^2 + ... + a^N + a^{N+1} \]

So
\[ aS_N = S_{N-1} + a^{N+1} \]

This gives
\[ (a-1)S_N = -1 + a^{N+1} \]
\[ S_N = \frac{1}{1-a} - \frac{a^{N+1}}{1-a}  \]

Taking the limit $N\rightarrow\infty$, we find that
\begin{align*}
\lim_{N\rightarrow\infty}S_N &= \frac{1}{1-a} - \lim_{N\rightarrow\infty} \frac{a^{N+1}}{1-a}
 &=
 \begin{cases}
 	\frac{1}{1-a} & \text{if } |a| < 1 \\
 	\text{DNE} & \text{if } |a| \geq 1
 \end{cases}
\end{align*}

Thus
\[ \sum_{n=0}^N a^n = \frac{1}{1-a} \text{ for } |a| < 1 \]
and
\[ \sum_{n=0}^N a^n \text{ diverges for } |a| \geq 1 \]

\end{document}