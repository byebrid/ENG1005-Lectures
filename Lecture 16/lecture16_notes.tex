\documentclass[11pt]{article}

\usepackage{amsmath}
\usepackage{amsfonts}
\usepackage{amssymb}

% Give ourself extra space for text
\usepackage[left = 2.2cm, right = 2.2cm, top = 1.8cm, bottom = 2.8cm]{geometry}

% Allows us to easily change the numbering system used in things like \begin{enumerate}. https://ctan.org/tex-archive/macros/latex/contrib/enumitem/
\usepackage[shortlabels]{enumitem}

% Turns table of contents, \refs, etc. into hyperlinks
\usepackage{hyperref}

% Common sets
\newcommand{\integers}{\mathbb{Z}}
\newcommand{\naturals}{\mathbb{N}}
\newcommand{\reals}{\mathbb{R}}

% Inverse hyperbolic functions
\DeclareMathOperator{\arcosh}{arcosh}
\DeclareMathOperator{\arsinh}{arsinh}
\DeclareMathOperator{\artanh}{artanh}

% I hat, J hat, K hat
\newcommand{\ihat}{\boldsymbol{\hat{\textbf{\i}}}}
\newcommand{\jhat}{\boldsymbol{\hat{\textbf{\j}}}}
\newcommand{\khat}{\boldsymbol{\hat{\textbf{k}}}}

% Better vectors (for single characters)
\renewcommand{\vec}[1]{\mathbf{#1}}

% Allows us to number equations in \begin{align} statements, etc.
\newcommand\numberthis{\addtocounter{equation}{1}\tag{\theequation}}

% Augmented matrices: this allows us to make augmented matrics using something like \begin{bmatrix}[cc|c]. Taken from Stefan Kottwitz at https://tex.stackexchange.com/questions/2233/whats-the-best-way-make-an-augmented-coefficient-matrix.
\makeatletter
\renewcommand*\env@matrix[1][*\c@MaxMatrixCols c]{%
  \hskip -\arraycolsep
  \let\@ifnextchar\new@ifnextchar
  \array{#1}}
\makeatother

% NOTE: This means \section does NOT number sections, but ensures that they appear in the table of contents, which does not occur if simply \section* is used. From egreg @ https://tex.stackexchange.com/a/30225.
\setcounter{secnumdepth}{0} % sections are level 1

\begin{document}
\title{ENG1005: Lecture 16}
\author{Lex Gallon}
\maketitle

\tableofcontents

\section*{Video link}
Click \href{https://echo360.org.au/lesson/G_8402119b-734b-4e1e-a3b4-7e907e86ddba_b944cecf-8ba5-40d3-a870-0243a0a9e78c_2020-04-28T15:58:00.000_2020-04-28T16:53:00.000/classroom#sortDirection=desc}{here} for a recording of the lecture.

\section{LU decomposition - continued}
\[ LU \vec{x} = \vec{b} \Leftrightarrow LU \vec{x} = \vec{b} 
\Leftrightarrow \left. \begin{cases} 
L \vec{y} = \vec{b} \\
U \vec{x} = \vec{y}
\end{cases} \right\}
(*)
\]

\section{Determining the LU decomposition}
Algorithm to perform LU decomposition:
\begin{enumerate}[ (i) ]
\item Perform elementary row operations 
\[ R_i \rightarrow R_i - \alpha_{ij} R_j \]
on $A$ until $A$ is in upper triangular form, which gives $U$.
\item Replace the zeros in the $(i, j)$ position of the identity matrix with the constants $\alpha_{ij}$ to obtain the lower triangular matrix $L$.
\end{enumerate}

This algorithm then guarantees that 
\[ A = LU \]
provided no row swaps are needed to find $U$.

\subsection{Remark}
For $n \times n$ matrices $A$ that require row swaps to be reduced into upper triangular form there always exists an $n \times n$ permutation matrix $P$ such that $PA$ has an $LU$ decomposition, that is
\[ PA = LU \]
So in this case, rather than solving
\[ A \vec{x} = \vec{b}, \]
we solve
\[ PA \vec{x} = P \vec{b} \]
using an $LU$ decomposition.

\subsection{Example}
Find an $LU$ decomposition for
\[ \begin{bmatrix}
3 & 1 & 5 \\
3 & 4 & 8 \\
6 & 11 & 3 \\
\end{bmatrix} \]

\begin{enumerate}[ (i) ]
\item
\[ R_2 \rightarrow R_2 - R_1,\ \alpha_{21} = 1  \]
\[ \begin{bmatrix}
3 & 1 & 5 \\
0 & 3 & 3 \\
6 & 11 & 3 \\
\end{bmatrix} \]

\[ R_3 \rightarrow R_3 - 2R_1,\ \alpha_{31} = 2 \]
\[ \begin{bmatrix}
3 & 1 & 5 \\
0 & 3 & 3 \\
0 & 9 & -7 \\
\end{bmatrix} \]

\[ R_3 \rightarrow R_3 - 3R_2,\ \alpha_{32}= 3 \]
\[ \begin{bmatrix}
3 & 1 & 5 \\
0 & 3 & 3 \\
0 & 0 & -16 \\
\end{bmatrix} \]

So
\[ U = \begin{bmatrix}
3 & 1 & 5 \\
0 & 3 & 3 \\
0 & 0 & -16 \\
\end{bmatrix} \]

\item
So start with the identity matrix, then fill out appropriate cells with $\alpha$ values from above (compare indices of $\alpha$s and you should be able to see what we've done).
\[
L = \begin{bmatrix}
1 & 0 & 0 \\
1 & 1 & 0 \\
2 & 3 & 1 \\
\end{bmatrix}
\]

\[ \begin{bmatrix}
3 & 1 & 5 \\
3 & 4 & 8 \\
6 & 11 & 3 \\
\end{bmatrix} 
= LU 
= \begin{bmatrix}
1 & 0 & 0 \\
1 & 1 & 0 \\
2 & 3 & 1 \\
\end{bmatrix}
\begin{bmatrix}
3 & 1 & 5 \\
0 & 3 & 3 \\
0 & 0 & -16 \\
\end{bmatrix}
\]
\end{enumerate}

\subsection{Example}
Solve
\begin{align*}
3x_1 + x_2 + 5x_3 &= 10, \\
3x_1 + 4x_2 + 8x_3 &= 19 \\
6x_1 + 11x_2 + 3x_3 &= 31 \\
\end{align*}
using $LU$ decomposition.

\subsection{Solution}
Write this as
\[ A \vec{x} = \vec{b} \]
where 
\[ A = \begin{bmatrix}
3 & 1 & 5 \\
3 & 4 & 8 \\
6 & 11 & 3 
\end{bmatrix},\ 
\vec{x} = \begin{bmatrix}
x_1 \\
x_2 \\
x_3
\end{bmatrix},\ 
\vec{b} = \begin{bmatrix}
10 \\
19 \\
31
\end{bmatrix} \]

Then by the previous example, we have
\[ LU \vec{x} = \vec{b} \]
where
\[
L = \begin{bmatrix}
1 & 0 & 0 \\
1 & 1 & 0 \\
2 & 3 & 1 \\
\end{bmatrix} \text{ and }
U = \begin{bmatrix}
3 & 1 & 5 \\
0 & 3 & 3 \\
0 & 0 & -16 \\
\end{bmatrix}
\]

Setting $\vec{y} = U \vec{x}$, and we solve 
\[ L \vec{y} = \vec{b} \Rightarrow \begin{bmatrix}
1 & 0 & 0 \\
1 & 1 & 0 \\
2 & 3 & 1 \\
\end{bmatrix}
\begin{bmatrix}
y_1 \\
y_2 \\
y_3
\end{bmatrix}
= \begin{bmatrix}
10 \\
19 \\
31
\end{bmatrix}
\]

It follows that
\begin{align*}
y_1 &= 10 \\
y_1 + y_2 &= 19 \\
2y_1  + 3y_2 + y_3 &= 31 \\
\Rightarrow y_1 &= 10 \\
y_2 &= 19 - y_1 = 19 - 10 = 9 \\
y_3 &= 31 - 2y_1 - 3y_2 = 31 - 2 \times 10 - 3 \times 9 = -16 \\
\Rightarrow \vec{y} &= \begin{bmatrix}
10 \\
9\\
-16
\end{bmatrix}
\end{align*}

We next solve
\[ U \vec{x} = \vec{y} \Rightarrow 
\begin{bmatrix}
3 & 1 & 5 \\
0 & 3 & 3 \\
0 & 0 & -16 \\
\end{bmatrix}
\begin{bmatrix}
x_1\\
x_2\\
x_3
\end{bmatrix}
= \begin{bmatrix}
10 \\
9\\
-16
\end{bmatrix}
\]

\begin{align*}
3x_1 + x_2 + 5x_3 &= 10\\
3x_2 + 3x_3 &= 9\\
-16x_3 &= -16\\
\Rightarrow x_3 &= 1\\
x_2 &= \frac{1}{3}(9 - 3x_3) = \frac{1}{3}(9 - 3) = 2\\
x_1 &= \frac{1}{3}(10 - x_2 - 5x_3) = \frac{1}{3}(10 - 2 - 5 \times 1) = 1
\end{align*}

So the solution to the system of linear equations is
\[ (x_1, x_2, x_3) = (1, 2, 1) \]

\section{Matrix inversion §5.4}
Consider now a linear system of equations
\[ A \vec{x} = \vec{b} \quad (*) \]
where $A$ is an $n \times n$ matrix. Suppose we can find an $n \times n$ matrix $B$ such that
\[ BA = \mathbb{I}_n \]
Then multiplying (*) on the left by $B$ would give
\begin{align*}
BA \vec{x} &= B\vec{b} \\
\mathbb{I}_n \vec{x} &= B\vec{b} \\
\vec{x} &= B\vec{b}
\end{align*}
This shows that (*) would have a unique solution given by 
\[ \vec{x} = B\vec{b} \]

Here, $B$ is the \textbf{inverse matrix} of $A$.

\subsection{Definition}
Given an $n \times n$ matrix $A$, we say that $A$ is invertible or \textbf{non-singular} if there exists an $n \times n$ matrix $B$ such that
\[ AB = BA = \mathbb{I}_n \]
The matrix $B$ is the \textbf{inverse matrix} of $A$ and it is denoted by $A^{-1}$. If no such $B$ exists, then $A$ is \textbf{singular} (or not invertible).\\
I.e.
\[ A A^{-1} = A^{-1}A = \mathbb{I}_n \]

\end{document}