\documentclass[11pt]{article}

\usepackage{amsmath}
\usepackage{amsfonts}
\usepackage{amssymb}

% Give ourself extra space for text
\usepackage[left = 2.2cm, right = 2.2cm, top = 1.8cm, bottom = 2.8cm]{geometry}

% Allows us to easily change the numbering system used in things like \begin{enumerate}. https://ctan.org/tex-archive/macros/latex/contrib/enumitem/
\usepackage[shortlabels]{enumitem}

% Turns table of contents, \refs, etc. into hyperlinks
\usepackage{hyperref}

% Common sets
\newcommand{\integers}{\mathbb{Z}}
\newcommand{\naturals}{\mathbb{N}}
\newcommand{\reals}{\mathbb{R}}

% Inverse hyperbolic functions
\DeclareMathOperator{\arcosh}{arcosh}
\DeclareMathOperator{\arsinh}{arsinh}
\DeclareMathOperator{\artanh}{artanh}

% I hat, J hat, K hat
\newcommand{\ihat}{\boldsymbol{\hat{\textbf{\i}}}}
\newcommand{\jhat}{\boldsymbol{\hat{\textbf{\j}}}}
\newcommand{\khat}{\boldsymbol{\hat{\textbf{k}}}}

% Better vectors (for single characters)
\renewcommand{\vec}[1]{\mathbf{#1}}

% Allows us to number equations in \begin{align} statements, etc.
\newcommand\numberthis{\addtocounter{equation}{1}\tag{\theequation}}

% NOTE: This means \section does NOT number sections, but ensures that they appear in the table of contents, which does not occur if simply \section* is used. From egreg @ https://tex.stackexchange.com/a/30225.
\setcounter{secnumdepth}{0} % sections are level 1

\begin{document}
\title{ENG1005: Lecture 11}
\author{Lex Gallon}
\maketitle

\tableofcontents

\section*{Video link}
Click \href{https://echo360.org.au/lesson/G_32340f5d-ff38-43d2-be9d-d88ddb1b3611_b944cecf-8ba5-40d3-a870-0243a0a9e78c_2020-04-08T14:58:00.000_2020-04-08T15:53:00.000/classroom#sortDirection=desc}{here} for a recording of the lecture.

\section{Vector products}
\subsection{Scalar (or inner or dot) product}
\begin{align*}
\vec{u} \cdot \vec{v} &= (u_1, u_2, u_3) \cdot (v_1, v_2, v_3) \\
&= u_1v_1 + u_2v_2 + u_3v_3 & \text{(a scalar)}
\end{align*}
This is equivalent to
\[ \vec{u} \cdot \vec{v} = |\vec{u}||\vec{v}| \cos(\theta) \]

Where $|\vec{u}|$ is known as the norm, length or magnitude of $\vec{u}$.
\[ |\vec{u}| = \sqrt{\vec{u} \cdot \vec{u}} = \sqrt{u_1^2 + u_2^2 + u_3^2} \]

\subsection{Cross product}
\begin{align*}
\vec{u} \times \vec{v} &= (u_1, u_2, u_3) \times (v_1, v_2, v_3) \\
&= (u_2v_3 - u_3v_2, u_3v_1 - u_1v_3, u_1v_2 - u_2v_1) \\ 
&= (u_2v_3 - u_3v_2)\ihat + (u_3v_1 - u_1v_3)\jhat + (u_1v_2 - u_2v_1)\khat
\end{align*}

\subsection{Geometric interpretation}
<MAYBE INSERT PICTURE HERE>
We can note that the cross product of 2 vectors $\vec{u}, \vec{v}$ is perpendicular to the plane on which both vectors lie.
\begin{enumerate}[ a) ]
\item $\vec{u} \times \vec{v} = |\vec{u}||\vec{v}| \sin(\theta)\vec{n}$, where $\vec{n}$ is that normal vector (and is a unit vector!).

Note that vectors are perpendicular/orthogonal ($\vec{u} \perp \vec{v}$) if and only if their dot product is \textbf{zero}. 

Also note that, in the case of the cross product, $\vec{n} \perp \vec{u}$ and $\vec{n} \perp \vec{v}$. It then also follows that $\vec{n} \cdot \vec{u} = \vec{n} \cdot \vec{v} = 0$.
\item $\vec{u} \times \vec{v} = 0 = \vec{u} \times \vec{v} \cdot \vec{v}$\hspace{2cm} $(\vec{u} \times \vec{v} \perp \vec{u}\ \&\ \vec{v})$
\item $|\vec{u} \times \vec{v}| = |\vec{u}||\vec{v}|\sin(\theta)$ <MAYBE INSERT PICTURE HERE OF PARALLELOGRAM>
\end{enumerate}

\subsection{Example}
Compute $\vec{u} \times \vec{v}$, where $\vec{u}=(1,2,3)$ and $\vec{v} = (4, -3, 2)$

\subsection{Solution}
\begin{align*}
\vec{u} \times \vec{v} &= (u_2v_3 - u_3v_2, u_3v_1 - u_1v_3, u_1v_2 - u_2v_1) \\ 
&= (2 \cdot 2 - 3 \cdot (-3), 3 \cdot 4 - 1 \cdot 2, 1 \cdot (-3) - 2 \cdot 4) \\
&= (13, 10, -11)
\end{align*}

You could then verify this is true by checking that:

$\vec{u} \times \vec{v} \cdot \vec{u} = 0$ \& $\vec{u} \times \vec{v} \cdot \vec{v} = 0$

\section{Lines §4.3.1}
\subsection{Parametric/vector equation}
<MAYBE INSERT PICTURE HERE>
Basically, if we have two vectors representing points, we have the vector from one point to the other being $\vec{w} = \vec{v} - \vec{u}$. \\
$\vec{r}(t) = \vec{u} + t(\vec{v} - \vec{u})$\\
So $\vec{r}(0) = \vec{u}, r(1)=\vec{v}$. This gives us some function $\vec{r}(t)$ that lies between $\vec{u}$ and $\vec{v}$ for $0 \leq t \leq 1$ (note that $t$ can be any real number if you want).

$l = \{ \vec{r}(t) | -\infty < t < \infty \}$ \hspace{2cm} (line)

\subsection{Example}
Find the parametric equation of the line passing through $(1, 2, 3)$ and $(-1, 3, -2)$.

\subsection{Solution}
Let $\vec{u} = (1, 2, 3)$ and $\vec{v} = (-1, 3, -2)$.\\
We then set
\[ \vec{w} = \vec{v} - \vec{u} = (-1, 3, -2) - (1, 2, 3) = (-2, 1, -5) \]
Thus the parametric equation of the line is
\[ \vec{r}(t) = \vec{u} + t\vec{w} = (1, 2, 3) + t(-2, 1, -5),\ t \in \reals \]
\[ \Leftrightarrow \]
\[ \vec{r}(t) = (1-2t, 2+t, 3-5t),\ t \in \reals\]

\subsection{Algebraic/Cartesian equation}
Let $\vec{r}(t) = \vec{u} + t\vec{w},\ t \in \reals$ parametrise a line $l$. Then
\begin{align*}
\vec{x} = (x_1, x_2, x_3) \in l &\Leftrightarrow \vec{x} = \vec{r}(t) \text{ for some } t \in \reals \\
&\Leftrightarrow (x_1, x_2, x_3) = (u_1, u_2, u_3) + t(w_1, w_2, w_3) \\
&\Leftrightarrow (x_1, x_2, x_3) = (u_1 + tw_1, u_2 + tw_2, u_3 + tw_3) \\
&\Leftrightarrow (x_1, x_2, x_3) = u_i + tw_i,\ i=1,2,3 \\
&\Leftrightarrow \frac{x_i - u_i}{w_i} = t, \ i=1,2,3\ (w_i \not= 0) \\
&\Leftrightarrow \frac{x_1 - u_1}{w_1} = \frac{x_2 - u_2}{w_2} = \frac{x_3 - u_3}{w_3}
\end{align*}
which is the algebraic equation of the line. This shows that we can determine the points that lie on a line $l$ passing through the vectors $\vec{u}$ and $\vec{v}=\vec{u}+\vec{w}$ by solving
\[ \frac{x_1 - u_1}{w_1} = \frac{x_2 - u_2}{w_2} = \frac{x_3 - u_3}{w_3} \]

\textbf{Note:} If $w_1 = 0$,
\[ x_1=u_1,\ \frac{x_2 - u_2}{w_2} = \frac{x_3 - u_3}{w_3} \]

\subsection{Example}
Find the algebraic equation of the line that passes through the points $(-1, 2, -1)$ and $(1, 1, 1)$.

\subsection{Solution}
Set $\vec{u} = (-1, 2, -1)$ and $\vec{v} = (1, 1, 1)$.\\
Then
\[ \vec{w} = \vec{v} - \vec{u} = (1, 1, 1) - (-1, 2, -1) = (2, -1, 2) \]
So the algebraic equation of the line is
\[ \frac{x+1}{2} = \frac{y-2}{-1} = \frac{z+1}{2} \]

\end{document}